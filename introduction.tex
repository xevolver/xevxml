\chapter{Introduction}\label{cap:intro}

%High-performance computing~(HPC) system architectures are getting more
%complicated and diversified. Due to the system complexity, performance
%optimizations specific to processor architectures, system
%configurations, compilers, and/or libraries, called
%\emph{system-specific optimizations}, are mandatory and becoming more
%important to exploit the potential of a particular system; an
%application code must be thoroughly optimized and specialized for one
%platform to achieve high performance.  As a result, one HPC application
%often needs to have multiple versions, each of which is optimized in a
%different way for adapting to a particular platform.  The diversity of
%system architectures increases the number of optimized versions required
%for performance portability across major platforms.  To make matters
%worse, popular platforms can change drastically over time, and thus an
%application might need to be optimized not only for current major
%platforms but also for future ones.  Accordingly, an increase in system
%complexity and diversity would force a programmer to further invest
%enormous time and effort for HPC application development and
%maintenance.

%The goal of the Xevolver project is to separate system-specific,
%system-aware performance optimizations from application codes. 

%XevXML has been developed to express any code modifications separately
%from an application code. Practical HPC application development is a
%team work of multiple programmers.

XevXML is a code transformation framework that allows users to define
their own code transformations, called \emph{user-defined code
transformations}.  It exposes an abstract syntax tree~(AST) to users so
that they can apply any transformations to the AST. Transformation rules
written in external files can be defined for individual systems,
compilers, libraries, and so on.  That is, code transformations
representing system-aware optimizations can be separated from an
application code.  Hence, to achieve high performance, the users no
longer need to specialize an application code for a particular platform.

XevXML assumes that an application code is annotated with a special
mark, using directives and/or comments, and transformations are applied
to the marked parts of the code.  Note that the mark indicates ``where
to transform,'' but does not indicate ``how to transform.''  The
transformation rules that indicate how to transform the code are defined
in external files. If system-aware code modifications are expressed as
code transformations, users can express system-awareness separately from
an application code.

As the name implies, XevXML employs eXtensible Markup Language~(XML) to
represent an AST, i.e.,~an internal representation of code structures
used by compilers.  XML is a widely-used data format, and various
XML-related technologies and tools have already been standardized and
matured.  Therefore, the users implement special code transformations by
using only those standard tools.  This chapter briefly describes an
overview of code transformation with XevXML.

\section{An overview of XevXML}
XevXML has so far been developed on top of the ROSE compiler
infrastructure~\cite{rose}.  XevXML provides the interconversion between
an ROSE AST and an XML AST.  XevXML converts a ROSE AST to an XML
representation of the AST, called an \emph{XML AST}. In XevXML, an XML
AST is exposed to users.  After user-defined transformations, the
transformed XML AST is again converted back to a ROSE AST so that ROSE
can unparse it to a C or Fortran code.

The interconversion is achieved by combining the following two commands,
\texttt{src2xml} and \texttt{xml2src}.

\begin{framed}
\begin{description}
 \item[NAME]~\\
	    \texttt{src2xml} -- source-to-xml translator

 \item[SYNOPSIS]~\\
	    \texttt{src2xml [OPTIONS] INPUT-FILE}

 \item[DESCRIPTION]~\\ \texttt{src2xml} converts a C or Fortran code
	    into an XML document. \texttt{src2xml} reads a code from the
	    input file given by the command-line argument, and prints
	    an output XML document to the standard
	    output. \texttt{src2xml} also accepts some of ROSE
	    command-line options such as \texttt{-rose:verbose}. A ROSE
	    command-line option \texttt{-rose:skip\_syntax\_check} is
	    automatically appended to the command-line options because
	    it is required for some Fortran90 codes.

 \item[EXAMPLES]~\\ \texttt{src2xml hello.c $>$ hello.xml}\\ This command
	    will read \texttt{hello.c} and output its AST as an XML
	    document to \texttt{hello.xml}.
\end{description}
\end{framed}

\begin{framed}
\begin{description}
 \item[NAME]~\\
	    \texttt{xml2src} -- xml-to-source translator

 \item[SYNOPSIS]~\\
	    \texttt{xml2src [OPTIONS]}

 \item[DESCRIPTION]~\\ \texttt{xml2src} converts an XML AST to a C or
	    Fortran code. Since the original language, C or Fortran, is
	    recorded in an XML AST, \texttt{xml2src} generates a code
	    written in the original language. \texttt{xml2xml} reads an
	    XML AST from the standard input, and prints the generated
	    code to the standard output. At present, command-line
	    options for \texttt{xml2src} are simply ignored.

 \item[EXAMPLES]~\\ \texttt{xml2src $<$ hello.xml $>$ hello-again.c}\\
	    This command will read \texttt{hello.xml} and output its
	    code to \texttt{hello-again.c}.
\end{description}
\end{framed}

An XML AST is exposed to users. Thus, the users can apply any
transformation to the AST. As an AST is represented as an XML document,
any XML-related technologies and tools are available for the
transformations. At present, XevXML employs XML Stylesheet Language
Transformation~(XSLT) as the low-level interface to express the
transformation rules of an XML AST.  AST transformation is what
compilers do internally for code transformation. Therefore, XevXML is
capable of implementing various code transformations.

XevXML can easily collaborate with ROSE.  ROSE already has various
features of code analyses and transformations to implement custom code
transformation programs in C++.  It must be painful if a user is
required to reimplement those features from scratch for XevXML. So
XevXML provides not only the above basic commands but also some C++
classes and functions, which are helpful to read and write XML ASTs, so
that code transformation programs developed with ROSE can handle XML
ASTs. Those classes and functions will be further described in
Chapter~\ref{chap:internal}.

If a code transformation is general enough and hence reusable in many
applications, it could be implemented using either ROSE or XevXML.
However, code transformations in practice could be application-specific,
system-specific, domain-specific, and even programmer-specific. If such
a code transformation program is implemented with ROSE, the user needs
to maintain the program in addition to his/her application code. In
XevXML, only transformation rules are defined declaratively, and code
transformations are performed using standard XML tools. So the user does
not need to develop his/her own program for applying the rules to
application codes.

%XevXML needs to be flexible enough to express varoius code
%transformations required in practical performance optimizations. 

%If the user is familiar with ROSE, he/she can develop a code
%transformation by using ROSE.

%and write the result as an XML document.



\section{XML elements and attributes}\label{sec:xml}
Let's get started with a simple example, ``Hello, World!'' in C.
\begin{framed}
\begin{src}
#include <stdio.h>

int main()
{
  printf("Hello, World!\n");
  return 0;
}
\end{src}
\end{framed}


Using the \texttt{src2xml} command, the above code is converted to an
AST of the following XML document.

\begin{framed}
\begin{src}
<?xml version="1.0" encoding="UTF-8"?>
<SgSourceFile filename="hello.c" language="2" format="2">
  <SgGlobal>
    <SgFunctionDeclaration name="main"  end_name="0" >
        <SgTypeInt/>
      <SgFunctionParameterList/>
      <SgFunctionDefinition>
        <SgBasicBlock>
          <SgExprStatement>
            <SgFunctionCallExp>
              <SgFunctionRefExp symbol="printf" />
              <SgExprListExp>
                <SgCastExp mode="0" >
                    <SgPointerType base_type="SgModifierType" >
                      <SgModifierType modifier="const" >
                        <SgTypeChar/>
                      </SgModifierType>
                      <SgTypeChar/>
                    </SgPointerType>
                  <SgStringVal value="Hello, World!\n" paren="1"/>
                </SgCastExp>
              </SgExprListExp>
            </SgFunctionCallExp>
          </SgExprStatement>
          <SgReturnStmt>
            <SgIntVal value="0"  string="0" />
          </SgReturnStmt>
        </SgBasicBlock>
      </SgFunctionDefinition>
<PreprocessingInfo pos="2"  type="6" >
#include &lt;stdio.h&gt;

</PreprocessingInfo>
    </SgFunctionDeclaration>
  </SgGlobal>
</SgSourceFile>
\end{src}
\end{framed}

The data format of XML ASTs is designed to simplify the interconversion
between ROSE ASTs and XML ASTs. In general, an XML document consists of
XML elements and their attributes. In XML ASTs, each XML element
corresponds to a ROSE AST node. XML attributes of an XML element are
used to keep the necessary information to restore the ROSE AST node.  An
XML AST looks like a text representation of a ROSE AST.  In other words,
XevXML provides another interface, XML, to handle ROSE AST nodes.

In the above XML document, the first line just indicates that the file
is written in XML.  The root node of an AST is the SgSourceFile element
in the second line. The SgSourceFile element represents the whole C
code.  The SgGlobal element in the third line is a child node of the
root node, and indicates the global scope of the C code. In the global
scope, the main function is declared and defined. In the function body,
the first statement is an expression statement, and the second statement
is a return statement. Comments and preprocessor information such as
\texttt{\#include $<$stdio.h$>$} are written as strings within the
PreprocessingInfo element.

XML attributes are used to restore ROSE AST nodes. For example, the
SgStringVal element corresponds to a ROSE AST node of StringVal
representing a string.  Thus, a string of \texttt{"Hello,
World!$\backslash{}$n"} is written as its value attribute. Similarly,
the SgFunctionRefExp element is a reference to the name of a function,
and the function name is given as the symbol attribute. Let's change
``printf'' is to ``puts'' by a text editor. Then, when the modified XML
AST is converted to a C code (by using the \texttt{xml2src} command),
the function call of printf will be changed to that of puts. This is a
simple example to show that, in XevXML, XML data transformation results
in AST transformation and thereby code transformation.




%In this way, each XML element in an XML AST corresponds to a ROSE AST
%node.

See the ROSE reference manual~\cite{rosemanual} to learn more about the
definition of each AST node.

\section{XML data transformation}\label{sec:xslt}

%XML is a widely-used data format, and there are many techniques and
%tools to transform XML data. For example, 

XML data are texts, and various tools are thus available to modify an
XML AST.  As shown in Section~\ref{sec:xml}, even a text editor can
modify an XML AST.  One may consider that, in the case of using a text
editor, modifying a C/Fortran code is much easier than modifying its XML
AST. So why don't we directly modify the code?  The answer is to avoid
specializing the code for a particular platform.

In many cases, system-aware code optimizations assuming a particular
target platform are necessary to exploit the system performance.  A
problem is that those optimizations are often harmful to the performance
of another platform.  A pragmatic approach is to maintain multiple
versions of a code, each of which is optimized for a different
platform. However, this results in degrading the maintainability and
making legacy application migration more painful.

%The modified AST can be converted back to a C or Fortran code.

XevXML has been developed to replace code modifications with
``mechanical transformations'' of an XML AST.  There are several
benefits of the replacement.  One important benefit is that the original
code is not necessarily specialized for a particular platform.  In other
words, system-awareness is separated from an application code.  This
will be helpful to avoid maintaining multiple versions of an application
code.

Another benefit is that expert knowledge about performance optimizations
can be expressed in a machine-usable way.  Basically, performance
optimizations are very intellectual tasks that are often done on a
case-by-case basis.  However, focusing on a particular case, there are
repetitive patterns in code modifications for performance
optimizations. Thus, the code modifications can be replaced with a
smaller number of mechanical code transformations.

The mechanical code transformations required instead of code
modifications could be application-specific, system-specific,
domain-specific, and even programmer-specific. Thus custom code
transformations are often needed for special demands of individual
cases.  Therefore, XevXML has been developed for users to define their
own code transformations in an easy way.

In XevXML, XSLT is employed to describe custom transformation rules of
XML ASTs at the lowest abstraction level\footnote{Several high-level
interfaces for definition of code transformation rules are also under
active development in the Xevolver project.  One of such interfaces will
be described in Chapter~\ref{chap:json}.}  In XSLT, XML data
transformations are themselves written in XML.  XSLT uses XPath
expressions~\cite{xpath} to define a pattern within a tree of XML
elements and attributes.  During the transformation process of XSLT,
every XML element is visited in a depth-first manner.  When a pattern is
found at an XML element, the XML element is altered based on the rule
associated with the pattern.

A simple XPath expression looks like a UNIX file path. In a UNIX file
system, files and directories organize a tree structure. A file path is
a text string to specify a location in the directory tree.  There are
two ways to point to the location of a file or a diretory.  One is an
absolute path, and the other is a relative path.  If the string of a
path starts with \texttt{/}, the path is an absolute path, otherwise it
is a relative path. An absolute path is the path to a file or a
directory from the root diretory. For example, the root directory is
expressed by \texttt{/}, its sub-directory named ``sub'' is expressed by
\texttt{/sub}, and a file named ``xfile'' that is located in the ``sub''
directory is expressed by \texttt{/sub/xfile}.  Note that \texttt{/}
represents the root direstory and is also used as a delimiting
character.  On the other hand, a relative path indicates the path from
the working directory where a user or an application is located. When
the working diretory is \texttt{/sub}, a relative path to a
\texttt{xfile} can be represented as \texttt{xfile}, \texttt{./xfile},
\texttt{../sub/xfile}, etc.  Of course, those relative paths point to
the same location bebause \texttt{.} and \texttt{..} denote the working
diretory and its parent directory, respectively.

As well as a UNIX file path, an XPath expression also points to a
location in an XML document.  For example, the root of an XML document
is denoted by \texttt{/}.  A pattern in XML data is described by a
combination of XPath expressions.


An example of XSLT rules for AST transformation are as follows.
\begin{framed}
\begin{src}
<?xml version="1.0" encoding="shift_jis"?>
<xsl:stylesheet version=�g1.0�h ...omitted...>

  <xsl:template match="/">
    <xsl:apply-templates/>
  </xsl:template>

  <xsl:template match="*">
    <xsl:copy>
      <xsl:copy-of select="@*"/>
      <xsl:apply-templates/>
    </xsl:copy>
  </xsl:template>

  <xsl:template match="SgForStatement">
    <xsl:if test=".//*=SgForStatement">
    startLoopNest(); /* inserted */
    </xsl:if>
    <xsl:copy>
      <xsl:copy-of select="@*"/>
      <xsl:apply-templates/>
    </xsl:copy>
    <xsl:if test=".//*=SgForStatement">
    endLoopNest(); /* inserted */
    </xsl:if>
  </xsl:template>
</xsl:stylesheet>
\end{src}
\end{framed}
The above XML file defines three rules, each of which is described
within the \texttt{xsl:template} element. Based on these rules, an XML
document is transformed to another XML document, called an output XML
document.

The first rule matches the root of an XML document. The rule just
invokes \texttt{$<$xsl::apply-templates/$>$} that by default dictates to
visit all the child nodes and apply appropriate rules to them.

The second rule matches every XML element of an XML document, because
its XPath expression is given by a wild-card operator, \texttt{*}. The
rule is applied to an element unless a more specific rule matches the
element.  This rule simply copies the element and its attributes to the
output XML document.  The rule is recursively invoked because it invokes
\texttt{$<$xsl::apply-templates/$>$}.

The third rule matches only an SgForStatement element. It checks if
another SgForStatement element exists in the subtree of the matched
element.  Only if it exists, text data are inserted before and after the
matched element.

If the above XSLT rules are applied to an XML AST, two function calls,
\texttt{startLoopNest()} and \texttt{endLoopNest()}, are inserted before
and after each nested loop, and a single loop is unchanged as shown
below.
\begin{framed}
\begin{src}
beginLoopNest(); /* inserted */
for(i=0;i<N;i++){
  for(j=0;j<M;j++){
    /* loop body 1 */
  }
}
endLoopNest();  /* inserted */

for(j=0;j<M;j++){
 /* loop body 2 */
}
\end{src}
\end{framed}
This is an example of text insertion based on code pattern matching.
Although this kind of rule is useful in practice, more advanced code
transformations can be achieved by writing XSLT rules because XSLT can
change the structure of an XML AST.  Chapter~\ref{chap:xslt} will
describe how to write XSLT rules for AST transformation.

%It is hence necessary to easily define such specific code
%transformations.

%it is not easy for standard programmers to develop special code
%translators using compiler tools such as ROSE.

\section{Summary}
To be described.
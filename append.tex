\chapter{XML elements and their attributes}\label{sec:attrib}

\section{Class hierarchy}
\begin{verbatim}
SgNode -+- SgStatement   -+- SgDeclarationStatement
        |
        +- SgExpression  -+- SgBinaryOp
        |                 |
        |                 +- SgUnaryOp
        +- SgType         |
        |                 +- SgValueExp
        +- SgSupport
\end{verbatim}
\section{Statements}
This section describes XML elements of \texttt{SgStatement} subclasses
currently supported by XevXML.

\subsection{XML elements}
All XML elements of \texttt{SgStatement} subclasses are described below.
If an XML element of a \texttt{SgStatement} subclass has its own XML
attributes, the attributes are also listed in the description.
\begin{longtable}[l]{l|p{10cm}}
 \hline
 XML element name & Description \\
 \hline\hline
 \endfirsthead
 \multicolumn{2}{l} {(continued)}\\
 \hline
 XML element name & Description \\
 \hline\hline
 \endhead
 \hline
 \multicolumn{2}{r} {(continue to next page)}\\
 \endfoot
 \hline
 \endlastfoot
 \multicolumn{2}{l}{Declarations}\\
 \hline
 \texttt{SgAsmStmt}& asm statement (not tested)\\
 & ~\texttt{code} (string): assembly code \\
 & ~\texttt{volatile} (int): 0 (not volatile) or 1 (volatile)\\
\texttt{SgAttributeSpecificationStatement}& attribute specification (Fortran only)\\
 & ~\texttt{kind} (int): See Section~\ref{sec:stmtatt}.\\
 %& ~~~\texttt{SgAttributeSpecificationStatement::attribute\_spec\_enum}\\
 \texttt{SgClassDeclaration}& declaration of a class or a structure\\
 & ~\texttt{name} (string): class name\\
 & ~\texttt{type} (int): 0 (class), 1 (struct), or 2 (union)\\
 \texttt{SgCommonBlock}& COMMON block (Fortran only)\\
 \texttt{SgContainsStatement}& CONTAINS statement (Fortran only)\\
 \texttt{SgDerivedTypeStatement}& derived type declaration (Fortran only)\\
 & ~\texttt{name} (string): derived type name\\
 & ~\texttt{type} (int): 0 (class), 1 (struct), or 2 (union)\\
 \texttt{SgEntryStatement}& ENTRY statement (Fortran only)\\
 & ~\texttt{name} (string): entry name\\
 & ~\texttt{result} (string): result variable name\\
 \texttt{SgEnumDeclaration}& enum declaration\\
 & ~\texttt{name} (string): enum name\\
 \texttt{SgEquivalenceStatement}& EQUIVALENCE statement (Fortran only)\\
 \texttt{SgFormatStatement}& FORMAT statement (Fortran only)\\
 \texttt{SgFortranIncludeLine}& INCLUDE (Fortran only)\\
 & ~\texttt{filename} (string): filename name\\
 \texttt{SgFunctionDeclaration}& function declaration \\
 & ~\texttt{name} (string): function name\\
 & ~\texttt{end\_name} (int): \texttt{set\_named\_in\_end\_statement()}\\
 \texttt{SgFunctionParameterList}& function parameter list\\
 \texttt{SgImplicitStatement}& IMPLICIT statement (Fortran only)\\
 \texttt{SgInterfaceStatement}& INTERFACE statement (Fortran only)\\
 & ~\texttt{name} (string): interface name\\
 & ~\texttt{type} (int): \texttt{SgInterfaceStatement::generic\_spec\_enum}\\
 \texttt{SgModuleStatement}& MODULE statement (Fortran only)\\
 & ~\texttt{name} (string): module name\\
 & ~\texttt{type} (int): 0 (class), 1 (struct), or 2 (union)\\
 \texttt{SgNamelistStatement}& NAMELIST statement (Fortran only)\\
% & ~\texttt{group} (string): group name\\
 \texttt{SgPragmaDeclaration}& pragma declaration\\
 \texttt{SgProcedureHeaderStatement}& procedure declaration (Fortran only)\\
 & ~\texttt{name} (string): procedure name\\
 & ~\texttt{kind} (int): subprogram kind\\
 & ~\texttt{result} (string): result variable name\\
 & ~\texttt{pure} (int): 0 (not pure) or 1 (pure)\\
 & ~\texttt{elemental} (int): 0 (not elemental) or 1 (elemental)\\
 & ~\texttt{recursive} (int): 0 (not recursive) or 1 (recursive)\\
 \texttt{SgProgramHeaderStatement}& program header (Fortran only)\\
 & ~\texttt{name} (string): program name\\
 & \texttt{elabel} (int): \texttt{set\_end\_numeric\_label()}\\
 \texttt{SgTypedefDeclaration}& typedef declaration\\
 & ~\texttt{name} (string): typedef name\\
 \texttt{SgUseStatement}& USE statement (Fortran only)\\
 & ~\texttt{name} (string): module name\\
 & ~\texttt{only} (int): 0 (no only option) or 1 (only option)\\
 \texttt{SgVariableDeclaration}& variable declaration \\
 & ~\texttt{name} (string): variable name\\
 & ~\texttt{bitfield} (int): bit field\\
 & ~\texttt{modifier} (int): modifier\\
 \hline
 \multicolumn{2}{l}{Other statements}\\
 \hline
 \texttt{SgAllocateStatement}& ALLOCATE statement (Fortran only)\\
 \texttt{SgArithmeticIfStatement}& arithmetic IF statement (Fortran only)\\
 \texttt{SgBackspaceStatement}& BACKSPACE statement (Fortran only)\\
 & \texttt{err} (int): \texttt{set\_err()}\\
 & \texttt{iostat} (int): \texttt{set\_iostat()}\\
 & \texttt{unit} (int): \texttt{set\_unit()}\\
 \texttt{SgBasicBlock}& basic block\\
 \texttt{SgBreakStmt}& break statement\\
 & \texttt{slabel} (string): Fortran \texttt{DO} string label\\
 \texttt{SgCaseOptionStmt}& case option of switch statement \\
 & ~\texttt{construct} (string): case construct name\\
 \texttt{SgClassDefinition}& class definition\\
 & \texttt{sequence} (int): 1 for \texttt{SEQUENCE} ketword\\
 & \texttt{private} (int):  1 for \texttt{PRIVATE} keyword\\
 & \texttt{abstract} (int): 1 for \texttt{ABSTRACT} keyword\\
 \texttt{SgCloseStatement}& CLOSE statement (Fortran only)\\
 & \texttt{err} (int): \texttt{set\_err()}\\
 & \texttt{iostat} (int): \texttt{set\_iostat()}\\
 & \texttt{status} (int): \texttt{set\_status()}\\
 & \texttt{unit} (int): \texttt{set\_unit()}\\
 \texttt{SgComputedGotoStatement}& computed GO TO statement (Fortran only)\\
 \texttt{SgContinueStmt}& continue statement\\
 & \texttt{slabel} (string): Fortran \texttt{DO} string label\\
 \texttt{SgDeallocateStatement}& DEALLOCATE statement (Fortran only)\\
 \texttt{SgDefaultOptionStmt}& default option of switch statement\\
 & ~\texttt{construct} (string): default construct name\\
 \texttt{SgDoWhileStmt}& do-while statement\\
 \texttt{SgElseWhereStatement}& ELSEWHERE statement (Fortran only)\\
 \texttt{SgEndfileStatement}& ENDFILE statement (Fortran only)\\
 & \texttt{err} (int): \texttt{set\_err()}\\
 & \texttt{iostat} (int): \texttt{set\_iostat()}\\
 & \texttt{unit} (int): \texttt{set\_unit()}\\
 \texttt{SgExprStatement}& statement of an expression\\
 \texttt{SgFlushStatement}& FLUSH statement (Fortran only)\\
 \texttt{SgForAllStatement}& FORALL statement (Fortran95)\\
 \texttt{SgForInitStatement}& initial condition of for statement\\
 \texttt{SgForStatement}& for statement\\
 \texttt{SgFortranDo}& DO statement (Fortran only)\\
 & \texttt{end} (int): \texttt{set\_has\_end\_statement()}\\
 & \texttt{slabel} (string): \texttt{set\_string\_label()}\\
 & \texttt{style} (int): \texttt{set\_old\_style()}\\
 & \texttt{elabel} (int): \texttt{set\_end\_numeric\_label()}\\
 \texttt{SgFunctionDefinition}& function definition\\
 \texttt{SgGlobal}& global scope\\
 \texttt{SgGotoStatement}& goto statement\\
 & \texttt{slabel} (string): string label\\
 & \texttt{nlabel} (string): numeric label\\
 \texttt{SgIfStmt}& if statement\\
 & \texttt{end} (int): \texttt{set\_has\_end\_statement()}\\
 & \texttt{then} (int):\texttt{set\_use\_then\_keyword()}\\
 & \texttt{elabel} (int):\texttt{set\_end\_numeric\_label()}\\
 & \texttt{ellabel} (int):\texttt{set\_else\_numeric\_label()}\\
 \texttt{SgInquireStatement}& INQUIRE statemtent (Fortran only)\\
 & *** TODO *** \\
 \texttt{SgLabelStatement}& label statement\\
 & \texttt{slabel} (string): string label\\
 & \texttt{nlabel} (string): numeric label\\
 \texttt{SgNullStatement}& empty statement (for(...){;})\\
 \texttt{SgNullifyStatement}& NULLIFY statement (Fortran only)\\
 \texttt{SgOpenStatement}& OPEN statement (Fortran only)\\
 & *** TODO *** \\
 \texttt{SgPrintStatement}& PRINT statement (Fortran only)\\
 & *** TODO *** \\
 \texttt{SgReadStatement}& READ statement (Fortran only)\\
 & *** TODO *** \\
 \texttt{SgReturnStmt}& return statement\\
 \texttt{SgRewindStatement}& REWIND statement (Fortran only)\\
 & *** TODO *** \\
 \texttt{SgStopOrPauseStatement}& STOP and PAUSE statements (Fortran only)\\
 & ~\texttt{type} (int): 0 (unknown), 1 (stop), or 2 (pause)\\
 \texttt{SgSwitchStatement}& switch statement\\
 & ~\texttt{slabel} (string): string label\\
 & ~\texttt{elabel} (string): end numeric label\\
 \texttt{SgWhereStatement}& WHERE statement (Fortran only)\\
 \texttt{SgWhileStmt}& while statement\\
 & \texttt{end} (int): \texttt{set\_has\_end\_statement()}\\
 & \texttt{slabel} (string): \texttt{set\_string\_label()}\\
 & \texttt{elabel} (string): \texttt{set\_end\_numeric\_label()}\\
 \texttt{SgWriteStatement}& WRITE statement (Fortran only)\\
 & \texttt{err} (int): \texttt{set\_err()}\\
 & \texttt{fmt} (int): \texttt{set\_format()}\\
 & \texttt{iostat} (int): \texttt{set\_iostat()}\\
 & \texttt{nml} (int): \texttt{set\_namelist()}\\
 & \texttt{rec} (int): \texttt{set\_rec()}\\
 & \texttt{unit} (int): \texttt{set\_unit()}\\
 \end{longtable}

\subsection{XML attributes}\label{sec:stmtatt}
All XML elements of \texttt{SgStatament} subclasses have \texttt{label}
attribute to keep the numeric label number because Fortran allows a
statement to have a numetic label.

In addition, XML elements of \texttt{SgDeclarationStatament} subclasses
have the following attributes for modifiers.
\begin{itemize}
 \item \texttt{declaration\_modifier}

       In an XML AST, only one of the following flags can be specified
       by a decimal number to indicate the kind of a declaration.
       \begin{description}
	\item[0x01]: unknown
	\item[0x02]: default
	\item[0x04]: friend
	\item[0x08]: typedef
%	\item[0x10]: export
       \end{description}
 \item \texttt{type\_modifier}

       Multiple flags might be combined by using logical OR. The value
       of this XML attribute (combination of the following flags) is
       written as a decimal number to indicate the type information.
       \begin{description}
	\item[0x00001]: unknown value (error)
	\item[0x00002]: unknown value (default)
	\item[0x00004]: restrict qualifier (for C/C++)
	\item[0x00008]: allocatable attribute specifier (for Fortran 90)
	\item[0x00010]: asynchronous attribute specifier (for Fortran 2003)
	\item[0x00020]: bind attribute specifier (for Fortran 2003)
	\item[0x00040]: data attribute specifier (for Fortran 77)
	\item[0x00080]: dimension attribute specifier (for Fortran 77)
	\item[0x00100]: intent(in) attribute specifier (for Fortran 90)
	\item[0x00200]: intent(out) attribute specifier (for Fortran 90)
	\item[0x00400]: intent(inout) attribute specifier (for Fortran 90)
	\item[0x00800]: intrinsic attribute specifier (for Fortran 90)
	\item[0x01000]: optional attribute specifier (for Fortran 90)
	\item[0x02000]: optional attribute specifier (for Fortran 90)
	\item[0x04000]: optional attribute specifier (for Fortran 90)
	\item[0x08000]: save attribute specifier (for Fortran 77)
	\item[0x10000]: target attribute specifier (for Fortran 90)
	\item[0x20000]: value attribute specifier (for Fortran 2003)
       \end{description}
 \item \texttt{cv\_modifier}

        Must be either const, volatile, or neither.
       \begin{description}
	\item[0x1]: unknown value (error)
	\item[0x2]: default value
	\item[0x4]: constant qualifier
	\item[0x8]: volatile qualifier
       \end{description}
 \item \texttt{access\_modifier}

       Only one of the following flags can be specified by a decimal
       number to indicate the accessibility.
       \begin{description}
	\item[0x01]: error value
	\item[0x02]: private access (local to class members)
	\item[0x04]: protected access (local to class members and membrs of derived classes)
	\item[0x08]: public access (access within enclosing namespace)
	\item[0x10]: default value (public access)
	\item[0x20]: fortran default value
       \end{description}
 \item \texttt{storage\_modifier}

       Only one of the following flags can be specified by a decimal
       number to location or properties of declarations.
       \begin{description}
	\item[0x01]: error value
	\item[0x02]: default value
	\item[0x04]: extern storage modifier
	\item[0x08]: static storage modifier
	\item[0x10]: auto storage value
	\item[0x20]: (not used)
	\item[0x40]: register storage modifier
	\item[0x80]: mutable storage modifier
       \end{description}

 \item \texttt{thread\_local}

       0 or 1. If 1 is given, the variable is thread-local.
\end{itemize}

The \texttt{$<$SgAttributeSpecificationStatement$>$} element requires
the \texttt{kind} attribute below.
 %& ~~~\texttt{SgAttributeSpecificationStatement::attribute\_spec\_enum}\\
\begin{itemize}
 \item \texttt{kind}

       One of the following values can be specified (Fortran).
       \begin{description}
	\item[0]: unknown
	\item[1]: private
	\item[2]: public
	\item[3]: allocatable
	\item[4]: asynchronous
	\item[5]: bind
	\item[6]: data
	\item[7]: dimension
	\item[8]: external
	\item[9]: intent
	\item[10]: intrinsic
	\item[11]: optional
	\item[12]: parameter
	\item[13]: pointer
	\item[14]: protected
	\item[15]: save
	\item[16]: target
	\item[17]: value
	\item[18]: volatile
       \end{description}
\end{itemize}
In the case of \texttt{kind=\"9\"}, the \texttt{intent} attribute is
required to be set to 600 (IN), 601 (OUT), or 602 (INOUT)

\section{Expressions}

%In the Sage III class library, expressions are represented with
%\texttt{SgExpression} subclasses. XevXML supports \texttt{SgExpression}
%subclasses in the table below.

This section describes XML elements of \texttt{SgExpression} subclasses
currently supported by XevXML.  XML elements of most
\texttt{SgExpression} subclasses have \texttt{paren} and \texttt{lvalue}
attributes whose values are integers.  Their values are given to
\texttt{SgExpression::set\_need\_paren()} and
\texttt{SgExpression::set\_lvalue()}, respectively.  If an XML element
of a \texttt{SgExpression} subclass has its own XML attributes, the
attributes are also listed in the description.

\begin{longtable}[l]{l|p{10cm}}
 \hline
 XML element name & Description \\
 \hline\hline
 \endfirsthead
 \multicolumn{2}{l} {(continued)}\\
 \hline
 XML element name & Description \\
 \hline\hline
 \endhead
 \hline
 \multicolumn{2}{r} {(continue to next page)}\\
 \endfoot
 \hline
 \endlastfoot
 \multicolumn{2}{l}{Binary operators}\\
 \hline
 \texttt{SgAddOp} & + operator (addition)\\
 \texttt{SgAndAssignOp} & \&= operator (bitwise AND assignment)\\
 \texttt{SgAndOp} & \&\& operator (logical AND)\\
 \texttt{SgArrowExp} & -$>$ operator (sturcture derefence (pointer))\\
 \texttt{SgAssignOp} & = operator (assignment)\\
 \texttt{SgBitAndOp} & \& operator (bitwise AND)\\
 \texttt{SgBitOrOp} & \texttt{|} operator (bitwise OR)\\
 \texttt{SgBitXorOp} & \^{} operator (bitwise XOR)\\
 \texttt{SgCommaOpExp} & , operator (comma)\\
 \texttt{SgConcatenationOp} & Fortran // operator (string concatenation)\\
 \texttt{SgDivAssignOp} & /= operator (division assignment)\\
 \texttt{SgDivideOp} & / operator (division)\\
 \texttt{SgEqualityOp} & == operator (equal to)\\
 \texttt{SgExponentiationOp} & Fortran ** operator (exponential) \\
 \texttt{SgGreaterOrEqualOp} & $>$= operator (greater than or equal to)\\
 \texttt{SgGreaterThanOp} & $>$ operator (greater than) \\
 \texttt{SgIorAssignOp} &  \texttt{|}= operator (bitwise OR assignment) \\
 \texttt{SgLessOrEqualOp} & $<$= operator (less than or equal to)\\
 \texttt{SgLessThanOp} & $<$ operator (less than)\\
 \texttt{SgLshiftAssignOp} & $<<$= operator (bitwise left shift assignment) \\
 \texttt{SgLshiftOp} & $<<$ operator (bitwise left shift) \\
 \texttt{SgMinusAssignOp} & -= operator (subtraction assignment) \\
 \texttt{SgModAssignOp} & \%= operator (modulo assignment)\\
 \texttt{SgModOp} & \% operator (modulo) \\
 \texttt{SgMultAssignOp} & *= operator (multiplication assignment)\\
 \texttt{SgMultiplyOp} & * operator (multiplication)\\
 \texttt{SgNotEqualOp} & != operator (not equal to)\\
 \texttt{SgOrOp} & \texttt{||} operator (logical OR)\\
 \texttt{SgPlusAssignOp} & += operator (addition assignment)\\
 \texttt{SgPointerAssignOp} & Fortran =$>$ operator (pointer assignment) \\
 \texttt{SgPntrArrRefExp} & array subscript (\texttt{a[b]})\\
 \texttt{SgRshiftAssignOp} & $>>$= operator (bitwise right shift assignment)\\
 \texttt{SgRshiftOp} & $>>$ operator (bitwise right shift)\\
 \texttt{SgSubtractOp} & - operator (subtraction)\\
 \texttt{SgUserDefinedBinaryOp} & user defined binary operator\\
 & ~\texttt{name} (string) :operator name \\
 \texttt{SgXorAssignOp} & \^{}= operator (bitwise XOR assignment)\\
 \hline
 \multicolumn{2}{l}{Unary operators}\\
 \hline
 \texttt{SgAddressOfOp} & \& operator (address)\\
 \texttt{SgBitComplementOp} & \textasciitilde operator (bitwise NOT)\\
 \texttt{SgMinusMinusOp} & \texttt{--} operator (decrement)\\
 & ~\texttt{mode} (int): 0 (prefix) or 1 (postfix)\\
 \texttt{SgMinusOp} & Unary - operator (minus)\\
 & ~\texttt{mode} (int): 0 (prefix) or 1 (postfix)\\
 \texttt{SgNotOp} & ! operator (logical NOT)\\
 \texttt{SgPlusPlusOp} & ++ operator (increment)\\
 & ~\texttt{mode} (int): 0 (prefix) or 1 (postfix)\\
 \texttt{SgUnaryAddOp} & Unary + operator (plus)\\
 \texttt{SgUserDefinedUnaryOp} & user defined unary operator\\
 & ~\texttt{name} (string) :operator name \\
 \hline
 \multicolumn{2}{l}{Values}\\
 \hline
 \texttt{SgBoolValExp}& boolean value \\
 & ~\texttt{value} (int) \\
 \texttt{SgCharVal}& character \\
 & ~\texttt{value} (char) \\
 & ~\texttt{string} (string): the value as a string \\
 \texttt{SgComplexVal}& complex number value \\
 \texttt{SgDoubleVal}& double-precision float value \\
 & ~\texttt{value} (double) \\
 & ~\texttt{string} (string): the value as a string \\
 \texttt{SgEnumVal}& enum value \\
 & ~\texttt{value} (int) \\
 \texttt{SgFloatVal}& float value \\
 & ~\texttt{value} (float) \\
 & ~\texttt{string} (string): the value as a string \\
 \texttt{SgIntVal}& integer value\\
 & ~\texttt{value} (int) \\
 & ~\texttt{string} (string): the value as a string \\
 \texttt{SgLongDoubleVal}& long double value\\
 & ~\texttt{value} (long double) \\
 & ~\texttt{string} (string): the value as a string \\
 \texttt{SgLongIntVal}& long int value \\
 & ~\texttt{value} (long int) \\
 & ~\texttt{string} (string): the value as a string \\
 \texttt{SgLongLongIntVal}& long long int value\\
 & ~\texttt{value} (long long int) \\
 & ~\texttt{string} (string): the value as a string \\
 \texttt{SgShortVal}& short int value\\
 & ~\texttt{value} (short int) \\
 & ~\texttt{string} (string): the value as a string \\
 \texttt{SgStringVal}& a string literal\\
 & ~\texttt{value} (string) \\
 & ~\texttt{single} (int): 0 (double quote) or 1 (single quote)\\
 \texttt{SgUnsignedCharVal}& unsigned character\\
 & ~\texttt{value} (unsigned char) \\
 & ~\texttt{string} (string): the value as a string \\
 \texttt{SgUnsignedIntVal}& unsigned int value\\
 & ~\texttt{value} (unsigned int) \\
 & ~\texttt{string} (string): the value as a string \\
 \texttt{SgUnsignedLongLongIntVal}& unsigned long long int value\\
 & ~\texttt{value} (unsigned long long int) \\
 & ~\texttt{string} (string): the value as a string \\
 \texttt{SgUnsignedLongVal}& unsigned long int value\\
 & ~\texttt{value} (unsigned long int) \\
 & ~\texttt{string} (string): the value as a string \\
 \texttt{SgUnsignedShortVal}& unsigned short int value\\
 & ~\texttt{value} (unsigned short int) \\
 & ~\texttt{string} (string): the value as a string \\
 \texttt{SgWcharVal}& whar\_t value\\
 & ~\texttt{value} (unsigned short) \\
 & ~\texttt{string} (string): the value as a string \\
 \hline
 \multicolumn{2}{l}{Other expressions}\\
 \hline
 \texttt{SgAggregateInitializer} & initialization with aggregated data (struct complex x=\{1,2\}) \\
 & ~\texttt{implicit} (int): 0 (need explicit braces) or 1 (not need)\\
 \texttt{SgAssignInitializer} & initialization (i=0)\\
 \texttt{SgAsteriskShapeExp} & Fortran * expression (write(*,*))\\
 \texttt{SgCastExp} & type cast ((int*)x)\\
 & ~\texttt{ctype} (int): cast type (\texttt{set\_cast\_type()}) \\
 & ~\texttt{implicit} (int): 0 (explicit cast) or 1 (implicit cast)\\
 \texttt{SgColonShapeExp} & Fortran colons for separating array dimensions (x(0:1))\\
 % This is for OpenCL vector initialization. e.g.  float2 x = float2(1,0f,2.0f);
 %\texttt{SgCompoundInitializer} & \\
 \texttt{SgConditionalExp} & ternary conditional (a?b:c)\\
 \texttt{SgConstructorInitializer} & initialization with a constructor (x=myclass(0,1))\\
 & ~\texttt{name} (int): \texttt{set\_need\_name()}. \\
 & ~\texttt{qual} (int): \texttt{set\_need\_qualifier()}.\\
 & ~\texttt{paren\_after\_name} (int): \texttt{set\_need\_parenthesis\_after\_name()}.\\
 & ~\texttt{unknown} (int): \texttt{set\_associated\_class\_unknown()}.\\
 \texttt{SgDotExp} & structure member access (x.y)\\
 \texttt{SgExprListExp} & list of expressions \\
 \texttt{SgFunctionCallExp} & function call  (f())\\
 \texttt{SgFunctionRefExp} & function reference\\
 & ~\texttt{name} (string): function name. \\
 & ~\texttt{kind} (int): 1(function(default)) or 2(subroutine) (for Fortran)\\
 \texttt{SgImpliedDo} & Fortran implied loop expression (\texttt{(/ /)})\\
 \texttt{SgLabelRefExp} & label refernce\\
 & ~\texttt{nlabel} (int): numeric label value\\
 & ~\texttt{type} (int): label type\\
 \texttt{SgNullExpression} & null expression \\
 \texttt{SgPointerDerefExp} & pointer dereference (*x, x is a pointer)\\
 \texttt{SgSizeOfOp} & sizeof operator (sizeof(int))\\
 \texttt{SgSubscriptExpression} & Fortran subscript expressions \\
 \texttt{SgVarArgEndOp} & end of vararg\\
 \texttt{SgVarArgOp} & variable length list of arguments \\
 \texttt{SgVarArgStartOp} & beginning of vararg\\
 \texttt{SgVarRefExp} & variable reference \\
 & ~\texttt{name} (string): variable name \\
\end{longtable}


\section{Types}

This section describes XML elements of \texttt{SgType} subclasses
currently supported by XevXML.  XML elements of most \texttt{SgType}
subclasses have the \texttt{kind} attribute whose value is an
integer. The attribute value is used as a type kind paremeter used in
Fortran.

\begin{longtable}[l]{l|p{10cm}}
 \hline
 XML element name & Description \\
 \hline\hline
 \endfirsthead
 \multicolumn{2}{l} {(continued)}\\
 \hline
 XML element name & Description \\
 \hline\hline
 \endhead
 \hline
 \multicolumn{2}{r} {(continue to next page)}\\
 \endfoot
 \hline
 \endlastfoot
 \texttt{SgArrayType}& array type\\
 \texttt{SgClassType}& class type\\
 \texttt{SgEnumType}& enum type\\
 \texttt{SgFunctionType}& function type\\
 \texttt{SgModifierType}& modifiers\\
 \texttt{SgPointerType}& pointer type\\
 \texttt{SgTypeBool}& boolean type\\
 \texttt{SgTypeChar}& character type\\
 \texttt{SgTypeComplex}& complex type (in C99, \texttt{double \_Complex x;})\\
 \texttt{SgTypeDefault}& unknown type (internal use)\\
 \texttt{SgTypeDouble}& double-precision floating-point type\\
 \texttt{SgTypeEllipse}& ellipse type (a variable number of parameters, ``\texttt{...}'')\\
 \texttt{SgTypeFloat}& single-precision floating-point type\\
 \texttt{SgTypeImaginary}& imaginary type (in C99, \texttt{float \_Imaginary x;})\\
 \texttt{SgTypeInt}& integer type\\
 \texttt{SgTypeLong}& long integer type\\
 \texttt{SgTypeLongDouble}& long double type\\
 \texttt{SgTypeLongLong}& long long integer type\\
 \texttt{SgTypeShort}& short integer type\\
 \texttt{SgTypeSignedChar}& signed character type\\
 \texttt{SgTypeSignedInt}& signed integer type\\
 \texttt{SgTypeSignedLong}& signed long integer type\\
 \texttt{SgTypeSignedLongLong}& signed long long integer type\\
 \texttt{SgTypeSignedShort}& signed short integer type\\
 \texttt{SgTypeString}& string type\\
 \texttt{SgTypeUnsignedChar}&unsigned character type\\
 \texttt{SgTypeUnsignedInt}&unsigned integer type\\
 \texttt{SgTypeUnsignedLong}&unsigned long integer type\\
 \texttt{SgTypeUnsignedLongLong}&unsigned long long integer type\\
 \texttt{SgTypeUnsignedShort}&unsigned short integer type\\
 \texttt{SgTypeVoid}&void type\\
 \texttt{SgTypedefType}&typedef type\\
\end{longtable}

\section{Other elements}
\begin{longtable}[l]{l|p{10cm}}
 \hline
 XML element name & Description \\
 \hline\hline
 \endfirsthead
 \multicolumn{2}{l} {(continued)}\\
 \hline
 XML element name & Description \\
 \hline\hline
 \endhead
 \hline
 \multicolumn{2}{r} {(continue to next page)}\\
 \endfoot
 \hline
 \endlastfoot
\texttt{SgDataStatementGroup}& a group of data statement objects (Fortran only)\\
\texttt{SgDataStatementObject}& data statement object (Fortran only)\\
\texttt{SgDataStatementValue}& data statement value (Fortran only)\\
 & ~\texttt{fmt} (int): \\
 & 0(unknown), 1(default), 2(explicit), 3(implicit), or 4(implied do)\\
\texttt{SgFormatItem}& item of FORMAT statement (Fortran only)\\
 & ~\texttt{fmt} (string): format data \\
 & ~\texttt{single} (int): non-zero to use single quotes \\
 & ~\texttt{double} (int): non-zero to use double quotes \\
\texttt{SgFunctionParameterTypeList}& type lists of function parameters\\
\texttt{SgNameGroup}& group of names for SgNamelistStatement (Fortran only)\\
 & ~\texttt{group} (string) : group name \\
 & ~\texttt{names} (string) : comma-separated names (e.g. "a,b,c,...") \\
\texttt{SgPragma}& prama in a pragma declaration\\
 & ~\texttt{pragma} (string): string(s) of a pragma \\
\texttt{SgSourceFile}& source file (root elemement of an XML AST)\\
 & ~\texttt{file} (string): filename of the original code \\
 & ~\texttt{lang} (int): 0 (error), 1 (unknown), 2 (C), 3 (C++), or 4 (Fortran)\\
 & ~\texttt{fmt} (int): 0 (unknown), 1 (fixed), or 2 (free) (for Fortran)\\
\texttt{SgTypedefSeq}& list of typedefs, for which the current SgType is the base type\\
\texttt{SgCommonBlockObject}& object in a COMMON block (Fortran only)\\
 & ~\texttt{name} (string): name of the COMMON block object \\
\texttt{SgInitializedName}& initialized name (declared symbols)\\
 & ~\texttt{name} (string): name of the symbol \\
 & ~\texttt{prev} (string): previous symbol for cray pointer\\
\texttt{SgInterfaceBody}& body of INTERFACE statement (Fortran only)\\
 & ~\texttt{name} (string): name of the INTERFACE function \\
 \texttt{SgRenamePair}& alias of a variable defined USE statement (Fortran only)\\
 & ~\texttt{lname} (string): local name\\
 & ~\texttt{uname} (string): use name\\
\end{longtable}

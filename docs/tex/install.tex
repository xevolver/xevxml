
\chapter{Installation}\label{chap:app}


\section{Requirements}

\begin{itemize}
 \item ROSE compiler infrastructure -- \url{http://rosecompiler.org/}
 \item Apache Xerces C++ 3.1.1 -- \url{http://xerces.apache.org/}
 \item Apache Xalan C++ 1.0 -- \url{http://xml.apache.org/xalan-c/}
 \item PicoJSON -- \url{https://github.com/kazuho/picojson/}
\end{itemize}




\section{Installation guide}
\begin{enumerate}
 \item First of all, some environment variables such as
       \texttt{LD\_LIBRARY\_PATH}, \texttt{JAVA\_HOME}, and \texttt{CXX}
       must be correctly set so as to use ROSE, Xerces, and Xalan.


 \item Create a new directory at the top diretocry for building the package.

       \vspace{5mm}
       \texttt{\% mkdir mybuild}

       \texttt{\% cd mybuild}
       \vspace{5mm}

 \item Run the \texttt{cmake} command at the created directory to
       generate Makefile and copy necessary files.

       \vspace{5mm}
       \texttt{\% cmake ../}
       \vspace{5mm}

       The \texttt{cmake} command accepts various options. For example,
       the install diretory is changed by using the
       \texttt{-DCMAKE\_INSTALL\_PREFIX} option.

       \vspace{5mm}
       \texttt{\% cmake -DCMAKE\_INSTALL\_PREFIX=/usr/local/xevxml ../}
       \vspace{5mm}

       If you need to use a specific version of a library or a header
       file, you can also use environment variables
       \texttt{CMAKE\_LIBRARY\_PATH} and \texttt{CMAKE\_INCLUDE\_PATH}.
       For example, if you need to use a library or a header file in
       \texttt{/home/user/local}, define those enviromnent variables as
       follows.

       \vspace{5mm}
       \texttt{\% export CMAKE\_LIBRARY\_PATH=/home/user/local/lib:\$CMAKE\_LIBRARY\_PATH}

       \texttt{\% export CMAKE\_INCLUDE\_PATH=/home/user/local/include:\$CMAKE\_INCLUDE\_PATH}
       \vspace{5mm}

      Those environment variables must be correctly set so that all of
      the necessary header files and libraries such as
      \texttt{picojson.h} and \texttt{rose.h} are found by the
       \texttt{cmake} command.

       See the \texttt{cmake} manual for more details~\cite{cmake}.

 \item Run the GNU make command.

   To compile the package,

       \vspace{5mm}
       \texttt{\% make}
       \vspace{5mm}

   To install the package,

       \vspace{5mm}
       \texttt{\% make install}
       \vspace{5mm}

   To test the package,

       \vspace{5mm}
       \texttt{\% make test}
       \vspace{5mm}

       Some of tests will be failed. But it does not necessarily mean
       that something is wrong with the built binaries. Even if
       everything goes well, XevXML fails in some tests. This is mainly
       because XevXML is built on top of ROSE and unable to pass the
       tests if ROSE cannot properly parse/unparse the test codes.

\end{enumerate}


\chapter*{Preface}

%  Purpose:
% \begin{itemize}
%    \item a. Welcome
%    \item b. Program features
%    \item c. Program benefits
% \end{itemize}
% \begin{center}
% *********************  \newline
% \end{center}
% \vspace{0.25in}


Welcome to the Xevolver XML introductory tutorial.  Xevolver
XML~(XevXML) is one of software products developed by the Xevolver
project.  The purpose of this project is to help migration of legacy HPC
applications to new systems by improving their performance portabilities
across system generations.  Since a high priority is given to
performance, an HPC application is often optimized and specialized for a
particular HPC system. As a result, the performance is not portable to
other systems.  To make matters worse, such system-specific code
optimizations are likely to be scattered over the application code. This
is one main reason why HPC application migration is so painful. It is
not affordable to reoptimize the whole code whenever a new system
becomes available.



XevXML is developed for XML-based AST transformations to provide an easy
way to user-defined code transformations.  The current implementation of
XevXML is built on top of the ROSE compiler framework. XevXML converts a
ROSE's AST to an XML document, and exposes it to programmers. So the
programmers can use any XML-related technologies and tools to transform
the AST. Then, the transformed AST is given back to the ROSE compiler
framework so that the AST is unparsed to generate a transformed
application code.  Instead of directly modifying an application code,
programmers can define their own code transformations to optimize the
code for each system.  System-specific optimizations are represented as
XML translation rules, which can be defined separately from an
application code.  This leads to separation between application
requirements and system requirements, expecting a lower migration cost
of HPC applications to new systems.

\chapter{AST Transformation Rules}\label{chap:xslt}

XevXML represents an AST in an XML format, and various XML tools are
hence available for AST transformation. In XevXML, a transformation rule
is (internally) represented as an XSLT rule. If a special code
transformation is needed, the most flexible and expressive way is to
write an XSLT rule for the transformation, even though high-level
interfaces are also developed for some specific purposes as described in
Chapter~\ref{chap:json}. An XSLT rule consists of XSLT template rules,
each of which is applied to an XML element if the element matches a
pattern of the XPath expression associated with the XSLT template rule.

XevXML provides some predefined XSLT template rules.  Users can
customize the predefined rules for their own purposes.  Thus, an XSLT
template rule can be used as a sample or a baseline for users to newly
define their special transformation rules.  This chapter describes how
to use and customize an XSLT rule for code transformation.


\section{Predefined Rules}\label{sec:predef}
Remember that, in XevXML, every transformation rule is internally
written in XSLT. You can write such a rule from scratch if you want.
But, a code transformation rule generally consists of many ``XSLT
template rules,'' and some of them are reusable in other code
transformation rules.  Therefore, XevXML offers some predefined XSLT
template rules that can be used as a part of a user-defined code
transformation rule.

Let's get started with a simple XSLT rule before explaining the
predefined rules.
\begin{framed}
\begin{src}
<?xml version="1.0" encoding="UTF-8"?>
<xsl:stylesheet version="1.0"
   xmlns:xsl="http://www.w3.org/1999/XSL/Transform"
   xmlns:exslt="http://exslt.org/common">

  <xsl:template match="/">
    <xsl:apply-templates/>
  </xsl:template>

  <xsl:template match="*">
    <xsl:copy>
      <xsl:copy-of select="@*"/>
      <xsl:apply-templates/>
    </xsl:copy>
  </xsl:template>

  <xsl:template match="SgFortranDo">
    <xsl:if test="preceding-sibling::*[1]/SgPragma/@pragma='xev loop_tag'">
      !pragma is found
    </xsl:if>
    <xsl:copy>
      <xsl:copy-of select="@*"/>
      <xsl:apply-templates mode="loopbody"/>
    </xsl:copy>
  </xsl:template>

  <xsl:template match="*" mode="loopbody">
    ! in a loop body
    <xsl:copy>
      <xsl:copy-of select="@*"/>
      <xsl:apply-templates mode="loopbody"/>
    </xsl:copy>
  </xsl:template>
</xsl:stylesheet>
\end{src}
\end{framed}

As mentioned in Section~\ref{sec:xslt}, an XSLT rule usually consists of
several ``template'' rules expressed by \texttt{$<$xsl:template$>$}
elements. In the above example, four template rules are defined. Each
template rule has a \texttt{match} attribute whose value is an XPath
expression. When an XSLT rule consisting of multiple template rules is
applied to an XML AST, every XML element in the XML AST, i.e.~every AST
node, is visited once in a depth-first manner. Then, if an XML element
matches the XPath expression given by the \texttt{match} attribute of a
template rule, the template rule is applied to the matched XML element.

In the above example, the XPath expression of the first template rule,
i.e.,~\texttt{/}, matches only the root node of an XML AST that is an
\texttt{$<$SgSourceFile$>$} element. Therefore, its rule is applied to
the root node, and \texttt{$<$xsl:apply-templates$>$} is called for
its child nodes.

The second template rule matches every node in the XML AST,
i.e.,~\texttt{*}, and its rule is applied to the node unless a more
specific rule matches the node.  This rule also calls
\texttt{$<$xsl:apply-templates$>$} for every child node.

The third template rule matches only an \texttt{$<$SgFortranDo$>$}
element. This uses an \texttt{$<$xsl:if$>$} element to check if the
XPath expression given by its \texttt{test} attribute is true. The
expression is true only if
\begin{itemize}
 \item The preceding sibling element of the \texttt{$<$SgFortranDo$>$}
       element has an \texttt{$<$SgPragma$>$} element as a child node,
 \item The \texttt{$<$SgPragma$>$} element has a \texttt{pragma}
       attribute, and
 \item The attribute value is a string of "xev loop\_tag".
\end{itemize}
If all the conditions are met, the template rule inserts a comment,
``pragma is found,'' to the code before copying sub-nodes and attributes
of the matched \texttt{$<$SgFortranDo$>$} element.

In the third template rule, an \texttt{$<$xsl::apply-templates$>$}
element is used with a \texttt{mode} attribute. As a result, template
rules with the same mode are applied to the child nodes. In this
example, the fourth template rule is applied to every child node of an
\texttt{$<$SgFortranDo$>$} element, for which all the above conditions
are met. The fourth template rule instead of the second rule matches
every node if \texttt{$<$xsl::apply-templates$>$} is used with the
\texttt{loopbody} mode. In this way, different template rules can be
applied only to a subset of AST nodes.


If the third rule and/or the fourth rule is modified so as to transform
a loop structure and/or a loop body, the above XSLT rule can be
considered as a loop transformation rule applied only to loops annotated
with \texttt{"xev loop\_tag"}. Although such a loop transformation rule
could be complex, its components, i.e.~XSLT template rules, are reusable
for many application codes.  Therefore, XevXML provides a library of
predefined XSLT template rules often required for basic loop
optimizations.


In the template rule library, a code transformation rule is assumed to
be comprised of three steps. One step is initialization, which usually
finds an annotation being used to indicate where to transform. Another
step is to move onto an XML element that is the root node of a subtree
to be transformed. The other step is to transform the subtree, and print
it out in XML. The library offers several predefined rules for each
step.

Using such predefined template rules for each step, a special directive
for unrolling Loop $i$ can be defined as follow.
\begin{framed}
 \begin{src}
<?xml version="1.0" encoding="UTF-8"?>
  <xsl:stylesheet version="1.0"
  xmlns:xsl="http://www.w3.org/1999/XSL/Transform">

  <xsl:import href="libCHiLL.xsl" />

  <xsl:output method="xml" encoding="UTF-8" />

  <xsl:template match="*" mode="xevInitHook">
   <xsl:apply-templates select="." mode="xevFindDirective">
      <xsl:with-param name="directiveName" select="'xev loop_tag'" />
    </xsl:apply-templates>
  </xsl:template>

  <xsl:template match="*" mode="xevMoveHook">
    <xsl:apply-templates select="." mode="xevGoToLoop">
      <xsl:with-param name="loopName" select="'i'" />
    </xsl:apply-templates>
  </xsl:template>

  <xsl:template match="*" mode="xevTransformationHook">
    <xsl:apply-templates select="." mode="chillUnroll">
      <xsl:with-param name="factor" select="4" />
      <xsl:with-param name="loopName" select="'i'" />
    </xsl:apply-templates>
  </xsl:template>
</xsl:stylesheet>
\end{src}
\end{framed}

The template rule with the \texttt{xevInitHook} mode represents the
first step of the initialization. In the rule,
\texttt{$<$xsl:apply-templates$>$} is used with the
\texttt{xevFindDirective} mode. This results in using a predefined
template rule associated with the \texttt{xevFindDirective} mode, and
also finds a directive, \texttt{xev loop\_tag}.

The template rule with the \texttt{xevMoveHook} mode represents the
second step. In the rule, the \texttt{xevGoToLoop} mode is used when
calling \texttt{$<$xsl:apply-templates$>$}. This results in visiting a
loop whose index variable is \texttt{i}. The loop is to be transformed
by the template rule of the third step.

The template rule with the \texttt{xevTransformationHook} mode
represents the third step. In the rule,
\texttt{$<$xsl:apply-templates$>$} is used with the
\texttt{chillUnroll} mode for loop unrolling. The
\texttt{chillUnroll} mode needs two parameters, \texttt{factor} and
\texttt{loopName}. In this example, the third rule unrolls Loop $i$ four
times.

In this way, we can organize a custom compiler directive by directly
using XevXML's low-level interface, i.e.,~XSLT.

In the current version of XevXML, all of predefined template rules are
written for Fortran programs. It is ongoing to define such rules for C
programs.  The predefined rules currently included in the library are as
follows.

\vspace{36pt}

\begin{longtable}[l]{l|l|l}
 \hline
 Mode & Parameters & Description \\
 \hline\hline
 \endfirsthead
 \multicolumn{3}{l} {(continued)}\\
 \hline
 Mode & Parameters & Description \\
 \hline\hline
 \endhead
 \hline
 \multicolumn{3}{r} {(continue to next page)}\\
 \endfoot
 \hline
 \endlastfoot
 \multicolumn{3}{l}{Initialization rules}\\
 \hline
\texttt{xevFindDirective} & \texttt{directiveName} & \multirow{2}{9cm}{A
 compiler directive specified by the \texttt{directiveName} attribute is
 found by this rule.}\\
&&\\ \hline
%\pagebreak
 \multicolumn{3}{l}{Movement rules}\\
 \hline

\texttt{xevGoToLoop} & \texttt{loopName} & \multirow{2}{9cm}{The
loop whose index variable is \texttt{loopName} is visited by using the rule.}\\
&&\\ \hline

\texttt{xevGoToVar} & \texttt{varName} & \multirow{2}{9cm}{A variable
 reference to \texttt{varName} is visited by using the rule.}\\
&&\\ \hline

\texttt{xevSkipToNthLoop} & \texttt{loopName} & \multirow{2}{9cm}{The
 $N$-th loop from a loop whose index variable is given by \texttt{loopName} is visited by using the rule.}\\
&\texttt{N} & \\ \hline

\texttt{xevGoToHere} & none & \multirow{2}{9cm}{This is
a dummy rule to do nothing at the second step of a code transformation.}\\
&&\\ \hline

%\pagebreak

 \multicolumn{3}{l}{Basic loop optimization rules}\\
 \hline

\texttt{xevLoopCollapse} & \texttt{firstLoop} & \multirow{3}{9cm}{Two
 loops are collapsed into a loop. Two parameters, \texttt{firstLoop} and
 \texttt{secondLoop}, specify the names of index variables of the loops
 to be collapsed. } \\
& \texttt{secondLoop} & \\ &&\\ \hline

% TODO xevLoopFission
%\texttt{xevLoopFission} & none & \multirow{3}{9cm}{A loop is broken into
% multiple loops. Now each statement in the original loop is moved to a
% different loop. Thus, every new loop contains only one statement in its
% body.} \\
%&&\\ &&\\ \hline

\texttt{xevLoopFusion} & none & \multirow{3}{9cm}{Two consecutive loops are
 fused into a single loop. Statements in the two loops are moved into
 the body of the new loop.} \\
 &&\\ &&\\ \hline

\texttt{xevLoopInterchange} & none & \multirow{1}{9cm}{Two
 consecutive loops are interchanged.} \\
\hline

\texttt{xevLoopInversion} & none & \multirow{2}{9cm}{A while loop body is
inverted into a do while loop with a if statement.} \\
&&\\ \hline

\texttt{xevLoopReversal} & none & \multirow{1}{9cm}{The order of a loop index
is reversed.} \\
\hline

\texttt{xevLoopSkewing} & none & \multirow{3}{9cm}{The index of the inner loop of
a loop nest is transformed into a new one that is dependent on the index of the outer loop.} \\
 &&\\ &&\\ \hline

\texttt{xevLoopStripMining} & \texttt{size} & \multirow{3}{9cm}{The iteration
of a loop is divided into two consecutive loops. \texttt{size} is
the division size of the loop index.} \\ &
 \texttt{factor} & \\ &&\\ \hline

\texttt{xevLoopTile} & \texttt{size1} & \multirow{3}{9cm}{
Two consecutive loops' iteration space is partitioned into blocks. \texttt{size1} is
 the block size of the outer loop and  \texttt{size2} is the block size of the inner loop.} \\
& \texttt{size2} &\\
&  &\\ \hline

\texttt{xevLoopUnroll} & \texttt{loopName} & \multirow{3}{9cm}{A loop is
 unrolled. \texttt{loopName} is the name of the index
 variable. \texttt{factor} is an unroll factor; every
 statement in the loop body is duplicated \texttt{factor} times.} \\ &
 \texttt{factor} & \\ &&\\ \hline

\texttt{xevLoopUnswitching} & none & \multirow{2}{9cm}{An conditional
if statement in a loop body is moved outside of the loop.} \\
&&\\ \hline

%\pagebreak

 \multicolumn{3}{l}{CHiLL-compatible versions of optimization rules}\\\hline

 \texttt{chillFuse} & none & \multirow{1}{9cm}{Two consecutive loops are
 fused into a single loop. Statements in the two loops are moved into
 the body of the new loop.} \\
 &&\\ &&\\ \hline

 \texttt{chillPermute} & \texttt{firstLoop} & \multirow{1}{9cm}{The order of up to
 three loops are changed. \texttt{firstLoop}, \texttt{secondLoop}, and
 \texttt{thirdLoop} are the names of index variables used by the loops
 to be permuted.} \\
 &\texttt{secondLoop}&\\ &\texttt{thirdLoop}&\\ \hline

% TODO chillSplit
% \texttt{chillSplit} & none & \multirow{1}{9cm}{This is the
%CHiLL-compatible version of \texttt{xevLoopFission}.} \\ \hline

\texttt{chillTile} & \texttt{size1} & \multirow{3}{9cm}{
Two consecutive loops' iteration space is partitioned into blocks. \texttt{size1} is
 the block size of the outer loop and  \texttt{size2} is the block size of the inner loop.} \\
& \texttt{size2} &\\
&  &\\ \hline

\texttt{chillUnroll} & \texttt{loopName}  & \multirow{3}{9cm}{A loop in
 unrolled. \texttt{loopName} is the name of the index
 variable. \texttt{factor} is an unroll factor; every
 statement in the loop body is duplicated \texttt{factor} times.} \\ &
 \texttt{factor} & \\ &&\\ \hline

\texttt{chillUnrollJam} & \texttt{loopName} & \multirow{4}{9cm}{The outer
 loop of a loop nest is unrolled. \texttt{loopName} is the name of the
 index variable. \texttt{factor} is an unroll factor; every statement in
 the loop body is duplicated \texttt{factor} times.} \\ &
 \texttt{factor} & \\ &&\\ &&\\  \hline
\end{longtable}

The above predefined template rules are basically designed under
assumption of the three steps defined with the \texttt{xevInitHook},
\texttt{xevMoveHook}, and \texttt{xevTransformHook} modes. But we do not
necessarily use the predefined rules for all of the three steps.  If
necessary, some steps can be manually described as in the first XSLT
example of this section. Indeed, the \texttt{xsltgen} command currently
uses predefined rules only for the third step.  The XSLT template rules
for the other two steps are written directly in individual XSLT
files. See the XSLT file generated by the \texttt{xsltgen} command for
details.

\section{Custom Rules}\label{sec:custom}

%In this section, suggested steps for implmenting custom XSLT rules are described.
%As an example, a rule of loop inversion[citation] is introduced.




%taking loop inversion~[citation] as an example. 


This section is a step-by-step tutorial to define a custom XSLT rule.

Since XSLT is an expressive programming language, we can define an
arbitrary XSLT template rule in various ways.  But there is a
frequently-used way to define an XSLT template rule for code
transformation.  One easy and typical way is summarized as follows.
\begin{description}
 \item[Step 1.] Write two versions of a code. One is the orginal version
	    and the other is its translated version. Let $C_\mathrm{in}$
	    and $C_\mathrm{out}$ be the original version and the
	    translated version, respectively. They can be considered as
	    an input code example and an output code example of the code
	    transformation to be defined below.

 \item[Step 2.] Convert $C_\mathrm{in}$ and $C_\mathrm{out}$ to their
	    XML representations, which are subtrees of XML ASTs, called
	    a $C_\mathrm{in}$ subtree and a $C_\mathrm{out}$ subtree,
	    respectively. The \texttt{src2xml} command can produce an
	    XML AST of a code. Thus, those subtrees should appear in the
	    XML AST if $C_\mathrm{in}$ and $C_\mathrm{out}$ are in the
	    code.

 \item[Step 3.] Write an XSLT template rule whose XPath expression
	    matches the code fragment to be transformed.  At this step,
	    the XSLT template rule can be empty, i.e.,~ the
	    \texttt{$<$xsl:template$>$} element does not have child
	    elements for writing XML elements into the output XML
	    data. A macthed code fragment will be completely removed if
	    the rule is empty.

 \item[Step 4.] Copy the $C_\mathrm{out}$ subtree into the XSLT template
	    rule. That is, the XML elements of the $C_\mathrm{out}$
	    subtree are simply used as child nodes of the
	    \texttt{$<$xsl:template$>$} element. At this step, any code
	    fragment matching the XSLT template rule is replaced with
	    $C_\mathrm{out}$.

 \item[Step 5.] Generalize the rule so that some statements and/or
	    expressions in the original code are copied to the
	    translated code. By comparing the subtrees of
	    $C_\mathrm{in}$ and $C_\mathrm{out}$, we can consider which
	    XML elements of $C_\mathrm{in}$ should be copied to the
	    subtree of the output code.
\end{description}


%If a code fragment is converted to an XML AST and copied inside an XSLT
%template rule, every subtree of an XML AST that matches the XPath
%expression of the XSLT template rule is replaced with the subtree in the
%template rule.

% TODO
% To be described.

%\iffalse
\subsubsection*{Step 1.~Write two versions of a code}
In the following, an XSLT rule to translate a simple \texttt{DO WHILE}
loop to its another version is defined as an example.  The two versions
are as follows.  This loop transformation is so-called loop
inversion~\cite{loopinv}.  Hereafter, the former version is called the
original loop, and the latter is the target loop.

\begin{framed}
\begin{src}
! original loop
do while(i<n)
  a(i) = 0
  i = i+1
end do
\end{src}
\end{framed}

\begin{framed}
\begin{src}
! target loop
if (i<n) then
  do
    a(i) = 0
    i = i+1
    if  (i<n)  cycle
    exit
  end do
endif
\end{src}
\end{framed}

%\subsection{Obtain Specific XML Elements of the Expected Code}
\subsubsection*{Step 2.~Convert the loops to XML}
%The following XML elements can be obtained with src2xml on the target code.

By converting the target loop, we can get the following XML data, which
is a subtree of an XML AST. If this subtree exists in an XML AST, the
target loop (as is) appears in the output code when the XML AST is
unparsed.

\begin{framed}
\begin{src}
<SgIfStmt end="1" then="1">
  <SgExprStatement>
    <SgLessThanOp>
      <SgVarRefExp name="i" />
      <SgVarRefExp name="n" />
    </SgLessThanOp>
  </SgExprStatement>
  <SgBasicBlock>
    <SgFortranDo style="0" end="1" slabel="">
      <SgNullExpression />
      <SgNullExpression />
      <SgNullExpression />
      <SgBasicBlock>
	<SgExprStatement>
          <SgAssignOp>
            <SgPntrArrRefExp lvalue="1">
              <SgVarRefExp name="a" />
              <SgExprListExp>
		<SgVarRefExp name="i" />
              </SgExprListExp>
            </SgPntrArrRefExp>
            <SgIntVal value="0" string="0" />
          </SgAssignOp>
	</SgExprStatement>
	<SgExprStatement>
          <SgAssignOp>
            <SgVarRefExp name="i" lvalue="1" />
            <SgAddOp>
              <SgVarRefExp name="i" />
              <SgIntVal value="1" string="1" />
            </SgAddOp>
          </SgAssignOp>
	</SgExprStatement>
	<SgIfStmt end="0" then="0">
          <SgExprStatement>
            <SgLessThanOp>
              <SgVarRefExp name="i" />
              <SgVarRefExp name="n" />
            </SgLessThanOp>
          </SgExprStatement>
          <SgBasicBlock>
            <SgContinueStmt />
          </SgBasicBlock>
          <SgBasicBlock />
	</SgIfStmt>
	<SgBreakStmt />
      </SgBasicBlock>
    </SgFortranDo>
  </SgBasicBlock>
  <SgBasicBlock />
</SgIfStmt>
\end{src}
\end{framed}

It is easy to get the XML data. If the original loop and the target loop
are in a code, their subtrees appear in the XML AST of the code.  The
XML AST can be easily obtained by using the \texttt{src2xml} command.
See Chapter~\ref{chap:intro} for details of the command.

%\subsubsection{Generalize /SgIfStmt/SgExprStatement}\label{subsubsec:cond}
%\shtodo{explanation}
%from

\subsubsection*{Step 3.~Write an initial XSLT template rule}

The syntax of an XSLT template rule is as follows.
\begin{framed}
\begin{src}
<xsl:template match=XXXX mode=ZZZZ>
   YYYY
</xsl:template>
\end{src}
\end{framed}
Here, XXXX and YYYY must be correctly written to define an XSLT template
rule.  Roughly speaking, XXXX indicates what kind of an XML element is
subject to this rule, and YYYY indicates how the XML element appears in
the output XML data.  For example, if we want to write a rule applied to
every \texttt{$<$SgWhileStmt$>$} element, \texttt{XXXX} should be
written as \texttt{"SgWhileStmt"}. If YYYY is empty, the matched XML
element and its sub-nodes do not appear in the output XML data. ZZZZ is
optinal but important for using predefined rules in XevXML. An XSLT
template rule associated with the ZZZZ mode does not match XXXX unless
it is invoked with the ZZZZ mode.

As shown in the first XSLT example in Section~\ref{sec:predef}, we can
write a rule so that it is applied only to an annotated code fragment.
An ``empty'' XSLT template rule applied only to \texttt{DO WHILE}
statements annotated with \texttt{!\$xev loop\_tag} is as follows.
\begin{framed}
\begin{src}
<?xml version="1.0" encoding="UTF-8"?>
<xsl:stylesheet version="1.0"
		xmlns:xsl="http://www.w3.org/1999/XSL/Transform"
		xmlns:exslt="http://exslt.org/common">

  <xsl:template match="/">
    <xsl:apply-templates/>
  </xsl:template>

  <xsl:template match="*">
    <xsl:copy>
      <xsl:copy-of select="@*"/>
      <xsl:apply-templates/>
    </xsl:copy>
  </xsl:template>

  <xsl:template match="SgWhileStmt">
    <xsl:choose>
      <xsl:when test="preceding-sibling::*[1]/SgPragma/@pragma='xev loop_tag'">
	<!-- rule for annotated statements should be written here -->
      </xsl:when>
      <xsl:otherwise>
	<xsl:copy>
	  <xsl:copy-of select="@*"/>
	  <xsl:apply-templates/>
	</xsl:copy>
      </xsl:otherwise>
    </xsl:choose>
  </xsl:template>
</xsl:stylesheet>
\end{src}
\end{framed}

Or, by using the predefined rules provided by XevXml, the above rule can
be simplified as follows.

\begin{framed}
\begin{src}
<?xml version="1.0" encoding="UTF-8"?>
<xsl:stylesheet version="1.0"
		xmlns:xsl="http://www.w3.org/1999/XSL/Transform"
		xmlns:exslt="http://exslt.org/common">

  <xsl:import href="libCHiLL.xsl" />

  <xsl:template match="*" mode="xevInitHook">
   <xsl:apply-templates select="." mode="xevFindDirective">
      <xsl:with-param name="directiveName" select="'xev loop_tag'" />
    </xsl:apply-templates>
  </xsl:template>

  <xsl:template match="*" mode="xevMoveHook">
    <xsl:apply-templates select="." mode="xevGoToLoop">
      <xsl:with-param name="loopName" select="'i'" />
    </xsl:apply-templates>
  </xsl:template>

  <xsl:template match="*" mode="xevTransformationHook">
    <xsl:apply-templates select="." mode="myLoopInversion"/>
  </xsl:template>

  <xsl:template match="SgWhileStmt" mode="myLoopInversion">
    <!-- rule for annotated statements should be written here -->
  </xsl:template>
</xsl:stylesheet>
\end{src}
\end{framed}

%Let's use the above rule as an initial rule and improve it in the
%following steps.  

Since the fourth rule is invoked with the \texttt{myLoopInversion} mode,
the rule is called the \texttt{myLoopInversion} rule.  The above
\texttt{myLoopInversion} rule will remove annotated \texttt{DO WHILE}
statements because it is empty (only a comment is written now) and no
XML element is hence output to the output XML data. Therefore, in the
following, the \texttt{myLoopInversion} rule is modified so that it
outputs a transformed version of the original loop. The other template
rules are unmodified.

\subsubsection*{Step 4.~Copy the subtree of the target loop to the XML template rule}

The comment in the \texttt{myLoopInversion} rule is replaced with the
subtree obtained at Step 2.  If this rule is applied to an XML AST,
every annotated \texttt{DO WHILE} statement with its loop body is
replaced with the target loop as is.

\begin{framed}
\begin{src}
<xsl:template match="SgWhileStmt" mode="myLoopInversion">
  <SgIfStmt end="1" then="1">
    <SgExprStatement> 
      <SgLessThanOp>
        <SgVarRefExp name="i" />
        <SgVarRefExp name="n" />
      </SgLessThanOp>
    </SgExprStatement>

    ... omitted ... (See Step 2)

  </SgIfStmt>
</xsl:template>
\end{src}
\end{framed}


\subsubsection*{Step 5.~Generalize the XSLT template rule}
The \texttt{myLoopInversion} rule in Step 4 simply replaces an annotated
\texttt{DO WHILE} loop with the target loop, and is usable only for loop
inversion of the original loop (See Step 1). Therefore, we need to
generalize the rule so that it can be reusable for loop inversion of
other loops.

To generalize it, we usually need to copy some statements and
expressions of the original loop to the target loop.  One is the
condition expression of the \texttt{DO WHILE} statement of the original
loop, i.e.,~\texttt{(i < n)}. If a \texttt{DO WHILE} statement has a
different condition expression, the expression should be used in the
target loop. Nonetheless, the \texttt{myLoopInversion} rule in Step 4
always uses \texttt{(i < n)} as the condition expression of the
transformed loop without repect to the condition expression of a
\texttt{DO WHILE} loop to be transformed.  Notice that the expression
appears twice in the target loop (See Step 1). One is the condition of
the first \texttt{IF} statement, and the other is the condition of the
second \texttt{IF} statement.

See the subtree of the target loop in Step 2. The
\texttt{myLoopInversion} rule in Step 4 is defined so as to output the
subtree as is, and thus to always output \texttt{(i < n)}, which
corresponds to the following \texttt{$<$SgExprStatement$>$} element and
its sub-nodes.
\begin{framed}
\begin{src}
...
<SgIfStmt end="1" then="1">
  <SgExprStatement>
    <SgLessThanOp>
      <SgVarRefExp name="i" />
      <SgVarRefExp name="n" />
    </SgLessThanOp>
  </SgExprStatement>
...
\end{src}
\end{framed}

On the other hand, the condition of a \texttt{DO WHILE} statement
corresponds to \texttt{SgWhileStmt/SgExprStatement} and its sub-nodes in
an XML AST\footnote{To figure out such a correspondence, it is helpful
to convert the original loop to its XML representation.}.  Since the
\texttt{myLoopInversion} rule matches an \texttt{$<$SgWhileStmt$>$}
element, the XPath expression of the condition is given by
\texttt{./SgExprStatement}~(relative path notation). Accordingly, the
rule should be modified so that the \texttt{$<$SgExprStatement$>$}
element specified by \texttt{./SgExpression} is copied to the output XML
data.
\begin{framed}
\begin{src}
...
<SgIfStmt end="1" then="1">
  <xsl:copy-of select="./SgExprStatement" />
...
\end{src}
\end{framed}

%\subsubsection{Generalize /SgIfStmt/SgBasicBlock/SgFortranDo/SgBasicBlock}
%\shtodo{explanation}

To make the \texttt{myLoopInversion} rule reusable for other loops, we
also need to copy all statements in the original loop body to the target
loop body. See the subtree of the target loop in Step 2.  Since the
original loop body contains two expression statements, \texttt{a(i)=0}
and \texttt{i=i+1}, the current \texttt{myLoopInvesion} rule always
outputs these two statements in the target loop body, and thus is not
reusable for other loops. The loop body is represented by the following
\texttt{$<$SgBasicBlock$>$} element and its sub-nodes.
\begin{framed}
\begin{src}
...
<SgBasicBlock>
  <SgExprStatement>
    <SgAssignOp>
      <SgPntrArrRefExp lvalue="1">
	<SgVarRefExp name="a" />
	<SgExprListExp>
	  <SgVarRefExp name="i" />
	</SgExprListExp>
      </SgPntrArrRefExp>
      <SgIntVal value="0" string="0" />
    </SgAssignOp>
  </SgExprStatement>
  <SgExprStatement>
    <SgAssignOp>
      <SgVarRefExp name="i" lvalue="1" />
      <SgAddOp>
	<SgVarRefExp name="i" />
	<SgIntVal value="1" string="1" />
      </SgAddOp>
    </SgAssignOp>
  </SgExprStatement>
</SgBasicBlock>
...
\end{src}
\end{framed}

On the other hand, the loop body of an \texttt{DO WHILE} statement is
represented by \texttt{SgWhileStmt/SgBasicBlock} and its sub-nodes.
Since the \texttt{myLoopInversion} rule matches an
\texttt{$<$SgWhileStmt$>$} element, the XPath expression of the loop
body is given by \texttt{./SgBasicBlock}~(relative path
notation). Accordingly, the \texttt{myLoopInversion} rule should be
modified so that all of the sub-nodes of the \texttt{$<$SgBasicBlock$>$}
element specified by \texttt{./SgBasicBlock} are copied to the output
XML data.

\begin{framed}
\begin{src}
...
<SgBasicBlock>
  <xsl:copy-of select="./SgBasicBlock/*" />
</SgBasicBlock>
...
\end{src}
\end{framed}

Finally, a generalized \texttt{myLoopInversion} rule is obtained as follows.
\begin{framed}
\begin{src}
<xsl:template match="SgWhileStmt" mode="myLoopInversion">
  <SgIfStmt end="1" then="1">
    <xsl:copy-of select="./SgExprStatement" />
    <SgBasicBlock>
      <SgFortranDo style="0" end="1" slabel="">
	<SgNullExpression />
	<SgNullExpression />
	<SgNullExpression />
	<SgBasicBlock>
	  <xsl:copy-of select="./SgBasicBlock/*" />
	  <SgIfStmt end="0" then="0">
	    <xsl:copy-of select="./SgExprStatement" />
	    <SgBasicBlock>
	      <SgContinueStmt />
	    </SgBasicBlock>
	    <SgBasicBlock />
	  </SgIfStmt>
	  <SgBreakStmt />
	</SgBasicBlock>
      </SgFortranDo>
    </SgBasicBlock>
    <SgBasicBlock />
  </SgIfStmt>
</xsl:template>
\end{src}
\end{framed}

The above rule is usable for loop inversion of not only the original
loop but also other standard \texttt{DO WHILE} loops.  In this way, we
can gradually and incrementally generalize an XSLT template rule step by
step so that the rule can cover a wider range of loops.  

\section{Summary}

In XevXML, XSLT template rules for basic loop transformations are
already predefined. The predefined rules will usually be used via the
\texttt{xsltgen} command described in Chapter~\ref{chap:json}.

In practical performance optimizations, some systems, applications,
application domains, and/or programmers may demand special code
transformations, which are not predefined, for high performance,
portability, maintainability, and so forth.  In such a case, they can
define their own code transformation rules for the special demands.

For defining a custom transformation rule, we first write a simple XSLT
rule usable only for replacing a particular code to its target code, and
then gradually generalize the rule so as to make it reusable for other
codes.  It is easy to write the initial simple rule by reference to an
XML AST of the target code.

%\shtodo{To be described}

%\fi

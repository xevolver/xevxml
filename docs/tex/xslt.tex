\chapter{AST Transformation Rules}\label{chap:xslt}

XevXML represents an AST in an XML format, and various XML tools are
hence available for AST transformation. In XevXML, a transformation rule
is (internally) represented as an XSLT rule. If a special code
transformation is needed, the most flexible and expressive way is to
write an XSLT rule for the transformation, even though high-level
interfaces are also developed for some specific purposes. An XSLT rule
consists of XSLT template rules, each of which is applied to an XML
element if the element matches a pattern on the XPath expression
associated with the XSLT template rule..

XevXML provides some predefined XSLT template rules.  Users can
customize the predefined rules for their own purposes.  Thus, an XSLT
template rule can be used as a sample or a baseline for users to newly
define their special transformation rules.  This chapter describes how
to define and customize an XSLT rule for code transformation.


\section{Predefined Rules}\label{sec:predef}
Remember that, in XevXML, every transformation rule is written in
XSLT. You can write such a rule from scratch if you want.  But, a code
transformation rule generally consists of many ``XSLT template rules,''
and some of them are reusable in other code transformation rules.
Therefore, XevXML offers some predefined XSLT template rules that can be
used as a part of a user-defined code transformation rule.

Let's get started with a simple XSLT rule before explaining the
predefined rules.
\begin{framed}
\begin{src}
<?xml version="1.0" encoding="UTF-8"?>
<xsl:stylesheet version="1.0"
   xmlns:xsl="http://www.w3.org/1999/XSL/Transform"
   xmlns:exslt="http://exslt.org/common">

  <xsl:template match="/">
    <xsl:apply-templates/>
  </xsl:template>

  <xsl:template match="*">
    <xsl:copy>
      <xsl:copy-of select="@*"/>
      <xsl:apply-templates/>
    </xsl:copy>
  </xsl:template>

  <xsl:template match="SgFortranDo">
    <xsl:if test="preceding-sibling::*[1]/SgPragma/@pragma='xev loop_tag'">
      !pragma is found
    </xsl:if>
    <xsl:copy>
      <xsl:copy-of select="@*"/>
      <xsl:apply-templates mode="loopbody"/>
    </xsl:copy>
  </xsl:template>

  <xsl:template match="*" mode="loopbody">
    ! in a loop body
    <xsl:copy>
      <xsl:copy-of select="@*"/>
      <xsl:apply-templates mode="loopbody"/>
    </xsl:copy>
  </xsl:template>
</xsl:stylesheet>
\end{src}
\end{framed}

As mentioned in Section~\ref{sec:xslt}, an XSLT rule usually consists of
several ``template'' rules expressed by \texttt{$<$xsl:template$>$}
elements. In the above example, four template rules are defined. Each
template rule has a \texttt{match} attribute whose value is an XPath
expression. When an XSLT rule consisting of multiple template rules is
applied to an XML AST, every XML element in the XML AST, i.e.~every AST
node, is visited once in a depth-first manner. Then, if an XML element
matches the XPath expression given by the \texttt{match} attribute of a
template rule, the template rule is applied to the matched XML element.

In the above example, the XPath expression of the first template rule,
i.e.,~\texttt{*}, matches only the root node of an XML AST that is an
\texttt{$<$SgSourceFile$>$} element. Therefore, its rule is applied to
the root node, and \texttt{$<$xsl:apply-templates$>$} is called for
its child nodes.

The second template rule matches every node in the XML AST, and its rule
is applied to the node unless a more specific rule matches the node.This
rule also calls \texttt{$<$xsl:apply-templates$>$} for every child
node.

The third template rule matches only an \texttt{$<$SgFortranDo$>$}
element. This uses an \texttt{$<$xsl:if$>$} element to check if the
XPath expression given by its \texttt{test} attribute is true. The
expression is true only if
\begin{itemize}
 \item The preceding sibling element of the \texttt{$<$SgFortranDo$>$}
       element has an \texttt{$<$SgPragma$>$} element as a child node,
 \item The \texttt{$<$SgPragma$>$} element has a \texttt{pragma}
       attribute, and
 \item The attribute value is a string of "xev loop\_tag".
\end{itemize}
If all the conditions are met, the template rule inserts a comment,
``pragma is found,'' to the code before copying sub-nodes and attributes
of the matched \texttt{$<$SgFortranDo$>$} element.

In the third template rule, an \texttt{$<$xsl::apply-templates$>$}
element is used with a \texttt{mode} attribute. As a result, template
rules with the same mode are applied to the child nodes. In this
example, the fourth template rule is applied to every child node of a
\texttt{$<$SgFortranDo$>$} element, for which all the above conditions
are met. The fourth template rule instead of the second rule matches
every node if \texttt{$<$xsl::apply-templates$>$} is used with the
\texttt{loopbody} mode. In this way, different template rules can be
applied only to a subset of AST nodes.


If the third rule and/or the fourth rule is modified to transform a loop
structure and/or a loop body, the above XSLT rule can be used as a loop
transformation rule applied only to loops annotated with \texttt{"xev
loop\_tag"}. Although such a loop transformation rule could be complex,
its components, i.e.~XSLT template rules, are reusable for many
application codes.  Therefore, XevXML provides a library of predefined
XSLT template rules often required for basic loop optimization techniques.


In the template rule library, a code transformation rule is assumed to
be comprised of three steps. One step is initialization, which usually
finds an annotation being used to indicate where to transform. Another
step is to move onto an XML element that is the root node of a subtree
to be transformed. The other step is to transform the subtree, and print
it out in XML. The library contains several predefined rules for each
step.

Using such predefined template rules for each step, a special directive
for unrolling Loop $i$ can be defined as follow.
\begin{framed}
 \begin{src}
<?xml version="1.0" encoding="UTF-8"?>
  <xsl:stylesheet version="1.0"
  xmlns:xsl="http://www.w3.org/1999/XSL/Transform">

  <xsl:import href="libCHiLL.xsl" />

  <xsl:output method="xml" encoding="UTF-8" />

  <xsl:template match="*" mode="xevInitHook">
   <xsl:apply-templates select="." mode="xevFindDirective">
      <xsl:with-param name="directiveName" select="'xev loop_tag'" />
    </xsl:apply-templates>
  </xsl:template>

  <xsl:template match="*" mode="xevMoveHook">
    <xsl:apply-templates select="." mode="xevGoToLoop">
      <xsl:with-param name="loopName" select="'i'" />
    </xsl:apply-templates>
  </xsl:template>

  <xsl:template match="*" mode="xevTransformationHook">
    <xsl:apply-templates select="." mode="chillUnroll">
      <xsl:with-param name="factor" select="4" />
      <xsl:with-param name="loopName" select="'i'" />
    </xsl:apply-templates>
  </xsl:template>
</xsl:stylesheet>
\end{src}
\end{framed}

The template rule with the \texttt{xevInitHook} mode represents the
first step of the initialization. In the rule,
\texttt{$<$xsl:apply-templates$>$} is used with the
\texttt{xevFindDirective} mode. This results in using a predefined
template rule associated with the \texttt{xevFindDirective} mode, and
also finds a directive, \texttt{xev loop\_tag}.

The template rule with the \texttt{xevMoveHook} mode represents the
second step. In the rule, \texttt{$<$xsl:apply-templates$>$} is used
with the \texttt{xevGoToLoop} mode. This results in visiting a loop
whose index variable is \texttt{i}. The loop is to be transformed by
this code transformation rule.

The template rule with the \texttt{xevTransformationHook} mode
represents the third step. In the rule,
\texttt{$<$xsl:apply-templates$>$} is used with the
\texttt{chillUnroll} mode for loop unrolling. The
\texttt{chillUnroll} mode needs two parameters, \texttt{factor} and
\texttt{loopName}. In this example, the third rule unrolls Loop $i$ four
times.

In this way, we can organize a custom compiler directive by directly
using XevXML's low-level interface, i.e.,~XSLT.

In the current version of XevXML, all of predefined template rules are
written for Fortran programs. It is ongoing to define such rules for C
programs.  The predefined rules currently included in the library are as
follows.

\vspace{36pt}

\begin{longtable}[l]{l|l|l}
 \hline
 Mode & Parameters & Description \\
 \hline\hline
 \endfirsthead
 \multicolumn{3}{l} {(continued)}\\
 \hline
 Mode & Parameters & Description \\
 \hline\hline
 \endhead
 \hline
 \multicolumn{3}{r} {(continue to next page)}\\
 \endfoot
 \hline
 \endlastfoot
 \multicolumn{3}{l}{Initialization rules}\\
 \hline
\texttt{xevFindDirective} & \texttt{directiveName} & \multirow{2}{9cm}{A
 compiler directive specified by the \texttt{directiveName} attribute is
 found by this rule.}\\
&&\\ \hline
%\pagebreak
 \multicolumn{3}{l}{Movement rules}\\
 \hline

\texttt{xevGoToLoop} & \texttt{loopName} & \multirow{2}{9cm}{The
loop whose index variable is \texttt{loopName} is visited by using the rule.}\\
&&\\ \hline

\texttt{xevGoToVar} & \texttt{varName} & \multirow{2}{9cm}{A variable
 reference to \texttt{varName} is visited by using the rule.}\\
&&\\ \hline

\texttt{xevSkipToNthLoop} & \texttt{loopName} & \multirow{2}{9cm}{The
 $N$-th loop from a loop whose index variable is given by \texttt{loopName} is visited by using the rule.}\\
&\texttt{N} & \\ \hline

\texttt{xevGoToHere} & none & \multirow{2}{9cm}{This is
a dummy rule to do nothing at the second step of a code transformation.}\\
&&\\ \hline

%\pagebreak

 \multicolumn{3}{l}{Basic loop optimization rules}\\
 \hline

\texttt{xevLoopCollapse} & \texttt{firstLoop} & \multirow{3}{9cm}{Two
 loops are collapsed into a loop. Two parameters, \texttt{firstLoop} and
 \texttt{secondLoop}, specify the names of index variables of the loops
 to be collapsed. } \\
& \texttt{secondLoop} & \\ &&\\ \hline

% TODO xevLoopFission
%\texttt{xevLoopFission} & none & \multirow{3}{9cm}{A loop is broken into
% multiple loops. Now each statement in the original loop is moved to a
% different loop. Thus, every new loop contains only one statement in its
% body.} \\
%&&\\ &&\\ \hline

\texttt{xevLoopFusion} & none & \multirow{3}{9cm}{Two consecutive loops are
 fused into a single loop. Statements in the two loops are moved into
 the body of the new loop.} \\
 &&\\ &&\\ \hline

\texttt{xevLoopInterchange} & none & \multirow{1}{9cm}{Two
 consecutive loops are interchanged.} \\
\hline

\texttt{xevLoopInversion} & none & \multirow{2}{9cm}{A while loop body is
inverted into a do while loop with a if statement.} \\
&&\\ \hline

\texttt{xevLoopReversal} & none & \multirow{1}{9cm}{The order of a loop index
is reversed.} \\
\hline

\texttt{xevLoopSkewing} & none & \multirow{3}{9cm}{The index of the inner loop of
a loop nest is transformed into a new one that is dependent on the index of the outer loop.} \\
 &&\\ &&\\ \hline

\texttt{xevLoopStripMining} & \texttt{size} & \multirow{3}{9cm}{The iteration
of a loop is divided into two consecutive loops. \texttt{size} is
the division size of the loop index.} \\ &
 \texttt{factor} & \\ &&\\ \hline

\texttt{xevLoopTile} & \texttt{size1} & \multirow{3}{9cm}{
Two consecutive loops' iteration space is partitioned into blocks. \texttt{size1} is
 the block size of the outer loop and  \texttt{size2} is the block size of the inner loop.} \\
& \texttt{size2} &\\
&  &\\ \hline

\texttt{xevLoopUnroll} & \texttt{loopName} & \multirow{3}{9cm}{A loop is
 unrolled. \texttt{loopName} is the name of the index
 variable. \texttt{factor} is an unroll factor; every
 statement in the loop body is duplicated \texttt{factor} times.} \\ &
 \texttt{factor} & \\ &&\\ \hline

\texttt{xevLoopUnswitching} & none & \multirow{2}{9cm}{An conditional
if statement in a loop body is moved outside of the loop.} \\
&&\\ \hline

%\pagebreak

 \multicolumn{3}{l}{CHiLL-compatible versions of optimization rules}\\\hline

 \texttt{chillFuse} & none & \multirow{1}{9cm}{Two consecutive loops are
 fused into a single loop. Statements in the two loops are moved into
 the body of the new loop.} \\
 &&\\ &&\\ \hline

 \texttt{chillPermute} & \texttt{firstLoop} & \multirow{1}{9cm}{The order of up to
 three loops are changed. \texttt{firstLoop}, \texttt{secondLoop}, and
 \texttt{thirdLoop} are the names of index variables used by the loops
 to be permuted.} \\
 &\texttt{secondLoop}&\\ &\texttt{thirdLoop}&\\ \hline

% TODO chillSplit
% \texttt{chillSplit} & none & \multirow{1}{9cm}{This is the
%CHiLL-compatible version of \texttt{xevLoopFission}.} \\ \hline

\texttt{chillTile} & \texttt{size1} & \multirow{3}{9cm}{
Two consecutive loops' iteration space is partitioned into blocks. \texttt{size1} is
 the block size of the outer loop and  \texttt{size2} is the block size of the inner loop.} \\
& \texttt{size2} &\\
&  &\\ \hline

\texttt{chillUnroll} & \texttt{loopName}  & \multirow{3}{9cm}{A loop in
 unrolled. \texttt{loopName} is the name of the index
 variable. \texttt{factor} is an unroll factor; every
 statement in the loop body is duplicated \texttt{factor} times.} \\ &
 \texttt{factor} & \\ &&\\ \hline

\texttt{chillUnrollJam} & \texttt{loopName} & \multirow{4}{9cm}{The outer
 loop of a loop nest is unrolled. \texttt{loopName} is the name of the
 index variable. \texttt{factor} is an unroll factor; every statement in
 the loop body is duplicated \texttt{factor} times.} \\ &
 \texttt{factor} & \\ &&\\ &&\\  \hline
\end{longtable}

\section{Custom Rules}\label{sec:custom}

In this section, suggested steps for implmenting custom XSLT rules are described.
As an example, a rule of loop inversion[citation] is introduced.

% TODO
To be described.

\iffalse

The rule will translate this original while loop
\begin{framed}
\begin{src}
    do while(i<n)
        a(i) = 0
        i = i+1
    end do
\end{src}
\end{framed}

to this target code.

\begin{framed}
\begin{src}
    if (i<n) then
        do
            a(i) = 0
            i = i+1
            if  (i<n)  cycle
            exit
        end do
    endif
\end{src}
\end{framed}



\subsection{Obtain Specific XML Elements of the Expected Code}

The following XML elements can be obtained with src2xml on the target code.

\begin{framed}
\begin{src}
<SgIfStmt end="1" then="1">
	<SgExprStatement>
		<SgLessThanOp>
			<SgVarRefExp name="i" />
			<SgVarRefExp name="n" />
		</SgLessThanOp>
	</SgExprStatement>
	<SgBasicBlock>
		<SgFortranDo style="0" end="1" slabel="">
			<SgNullExpression />
			<SgNullExpression />
			<SgNullExpression />
			<SgBasicBlock>
				<SgExprStatement>
					<SgAssignOp>
						<SgPntrArrRefExp lvalue="1">
							<SgVarRefExp name="a" />
							<SgExprListExp>
								<SgVarRefExp name="i" />
							</SgExprListExp>
						</SgPntrArrRefExp>
						<SgIntVal value="0" string="0" />
					</SgAssignOp>
				</SgExprStatement>
				<SgExprStatement>
					<SgAssignOp>
						<SgVarRefExp name="i" lvalue="1" />
						<SgAddOp>
							<SgVarRefExp name="i" />
							<SgIntVal value="1" string="1" />
						</SgAddOp>
					</SgAssignOp>
				</SgExprStatement>
				<SgIfStmt end="0" then="0">
					<SgExprStatement>
						<SgLessThanOp>
							<SgVarRefExp name="i" />
							<SgVarRefExp name="n" />
						</SgLessThanOp>
					</SgExprStatement>
					<SgBasicBlock>
						<SgContinueStmt />
					</SgBasicBlock>
					<SgBasicBlock />
				</SgIfStmt>
				<SgBreakStmt />
			</SgBasicBlock>
		</SgFortranDo>
	</SgBasicBlock>
	<SgBasicBlock />
</SgIfStmt>
\end{src}
\end{framed}

\subsubsection{Generalize /SgIfStmt/SgExprStatement}\label{subsubsec:cond}

\shtodo{explanation}

from

\begin{framed}
\begin{src}
<SgIfStmt end="1" then="1">
	<SgExprStatement>
		<SgLessThanOp>
			<SgVarRefExp name="i" />
			<SgVarRefExp name="n" />
		</SgLessThanOp>
	</SgExprStatement>
\end{src}
\end{framed}

to

\begin{framed}
\begin{src}
		<xsl:copy-of select="SgExprStatement" />
\end{src}
\end{framed}

\subsubsection{Generalize /SgIfStmt/SgBasicBlock/SgFortranDo/SgBasicBlock}

\shtodo{explanation}

from

\begin{framed}
\begin{src}
				<SgExprStatement>
					<SgAssignOp>
						<SgPntrArrRefExp lvalue="1">
							<SgVarRefExp name="a" />
							<SgExprListExp>
								<SgVarRefExp name="i" />
							</SgExprListExp>
						</SgPntrArrRefExp>
						<SgIntVal value="0" string="0" />
					</SgAssignOp>
				</SgExprStatement>
				<SgExprStatement>
					<SgAssignOp>
						<SgVarRefExp name="i" lvalue="1" />
						<SgAddOp>
							<SgVarRefExp name="i" />
							<SgIntVal value="1" string="1" />
						</SgAddOp>
					</SgAssignOp>
				</SgExprStatement>
\end{src}
\end{framed}

to

\begin{framed}
\begin{src}
					<xsl:copy-of select="SgBasicBlock/*" />
\end{src}
\end{framed}

\subsubsection{Generalize /SgIfStmt/SgBasicBlock/SgFortranDo/SgBasicBlock/SgIfStmt/SgExprStatement}

The generalization of the SgExprStatement in the loop body is almost the same as \ref{subsubsec:cond}

from

\begin{framed}
\begin{src}
					<SgExprStatement>
						<SgLessThanOp>
							<SgVarRefExp name="i" />
							<SgVarRefExp name="n" />
						</SgLessThanOp>
					</SgExprStatement>
\end{src}
\end{framed}

to

\begin{framed}
\begin{src}
		<xsl:copy-of select="SgExprStatement" />
\end{src}
\end{framed}


\subsection{Obtain Generalized Custom Rules}

\shtodo{explanation}


\begin{framed}
\begin{src}
<xsl:template match="SgWhileStmt" mode="xevLoopInversion">
	<SgIfStmt>
		<xsl:copy-of select="SgExprStatement" />
		<SgBasicBlock>
			<SgFortranDo style="0" end="1" slabel="">
				<SgNullExpression />
				<SgNullExpression />
				<SgNullExpression />
				<SgBasicBlock>
					<xsl:copy-of select="SgBasicBlock/*" />
					<SgIfStmt end="0" then="0">
						<xsl:copy-of select="SgExprStatement" />
						<SgBasicBlock>
							<SgContinueStmt />
						</SgBasicBlock>
						<SgBasicBlock />
					</SgIfStmt>
					<SgBreakStmt />
				</SgBasicBlock>
			</SgFortranDo>
		</SgBasicBlock>
		<SgBasicBlock />
	</SgIfStmt>
</xsl:template>
\end{src}
\end{framed}


As a result, the above rule can be obtained for loop inversion transformation.


\section{Summary}

\shtodo{To be described}

\fi

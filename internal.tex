\chapter{Internal Structures and Behaviors}\label{chap:internal}

Since ROSE already has various features of code analyses and
transformations useful to implement custom code transformation programs,
XevXML provides some C++ classes and functions, which are helpful for
such programs to read and write XML ASTs.  As a result, code
transformation programs developed with ROSE can handle XML ASTs.

This chapter describes XevXML classes and functions that allows ROSE to
handle XML ASTs. By inheriting the classes, users can develop customized
versions of \texttt{src2xml} and \texttt{xml2src} commands. For example,
\texttt{src2xml} would be customized so that some additional XML
attributes are written in an XML AST.


\section{Utility functions}
%A simplified version of the code is as follows.

%A minimum code of a ROSE-based translator is as follows.
By using ROSE, it is easy to write an identity translator, i.e.,~the
simplest translator.
\begin{framed}
\begin{src}
/* identity.cpp */
#include <rose.h>

int main(int argc,char** argv){
  SgProject* sageProject=frontend(argc,argv);

  /* do something here for code transformation */

  return backend(sageProject);
}
\end{src}
\end{framed}

The \texttt{frondend()} function reads a code and builds its AST. The
\texttt{backend()} function passes the AST to the backend for unparsing
and compilation.

XevXML offers a utility function, \texttt{XevConvertRoseToXml()}, for
such a ROSE-based translator to print out a ROSE AST as an XML AST. All
of the functions and classes provided XevXML are defined within a name
space, \texttt{XevXml}.


For example, a simplified version of the \texttt{src2xml} command is as
follows. It just prints out an XML AST, and then immediately ends.

\begin{framed}
\begin{src}
/* simple.cpp */
#include <iostream>
#include <rose.h>
#include <xevxml.hpp>

int main(int argc,char** argv){
  SgProject* sageProject=frontend(argc,argv);

  /* print out an XML AST to the standard output */
  XevXml::XevInitialize();
  XevXml::XevConvertRoseToXml(std::cout,&sageProject);
  XevXml::XevFinalize();

  /* return backend(sageProject); */
  return 0; /* backend is not called in this example */
}
\end{src}
\end{framed}

Run your compiler with appropriate options to specify the paths to
necessary header files and libraries.

\vspace{5mm}
\texttt{\% g++ -I... simple.cpp -o simple -L... -lxevxml -lrose -lxalan-c -lxerces-c }
\vspace{5mm}

\noindent The generated executable, \texttt{simple}, will print out an
XML AST of a C or Fortran code to the standard output.

\vspace{5mm}
\texttt{\% ./simple input.c}
\begin{verbatim}
<?xml version="1.0" encoding="UTF-8"?>
<SgSourceFile file="input.c" lang="2" fmt="2">
 <SgGlobal>
 ...omitted...
 </SgGlobal>
</SgSourceFile>
\end{verbatim}
\vspace{5mm}

A simplified version of the \texttt{xml2src} command is as follows.

\begin{framed}
\begin{src}
/* simple2.cpp */
#include <iostream>
#include <rose.h>
#include <xevxml.hpp>

int main(int argc,char** argv){
  SgProject* sageProject=0;

  /* read an XML AST from the standard input */
  XevXml::XevInitialize();
  if( XevXml::XevConvertXmlToRose(std::cin,&sageProject) == false ){
    std::cerr << " failed" << std::endl;
    abort();
  }
  else {
    /* check the AST (optional) */
    AstTests::runAllTests(sageProject);
    /* unparse the AST and print it out */
    XevXml::XevUnparseToStream(std::cout,&sageProject);
  }
  XevXml::XevFinalize();

  return 0;
}
\end{src}
\end{framed}

The \texttt{XevConvertXmlToRose()} function reads an XML AST and
converts it to a ROSE AST. If the conversion does not fail, the
\texttt{XevUnparseToStream()} function is called to print the ROSE AST
to a C++ stream.  Accordingly, the above code, \texttt{simple2}, reads
an XML AST from the standard input, and then unparses it. The unparsed
code is printed out to the standard output.

The following combination of the above two commands works as an identity
translator.

\vspace{5mm}
\texttt{\% ./simple input.c | ./simple2 }
\begin{verbatim}
#include <stdio.h>

int main()
{
  ...omitted...
  return 0;
}
\end{verbatim}


The behaviors of the \texttt{XevConvertRoseToXml()} function can be
changed using a \texttt{XevXmlOption} class object. For example, a
slightly-modified vetsion of \texttt{simple.cpp} that uses the
\texttt{XevXmlOption} is as follows.
\begin{framed}
\begin{src}
/* simple-mod.cpp */
#include <iostream>
#include <rose.h>
#include <xevxml.hpp>

int main(int argc,char** argv){
  SgProject* sageProject=frontend(argc,argv);

  /* print out an XML AST to the standard output */
  XevXml::XevInitialize();
  XevXml::XevXmlOption opt;
  /* the following options are disabled by default */
  opt.getFortranPragmaFlag() = true; // parse Fortran pragmas
  opt.getPrintAddressFlag() = true; // print the address of every AST node

  /* execute the conversion with the above options */
  XevXml::XevConvertRoseToXml(std::cout,&sageProject,&opt);
  XevXml::XevFinalize();

  return 0;
}
\end{src}
\end{framed}

In the above code, all the options disabled by default are enabled.  By
default, every ``xev pragma'' in a Fortran code, which starts with
\texttt{!\$xev}, is handled as a comment.  By enabling the flag accessed
by the \texttt{getFortranPragmaFlag()} method, such a pragma will be
converted to an XML element. The other options are basically used for
debugging.

\section{Visitor classes}

As described in the previous section, a C code whose file name is
\texttt{input.c} will be converted to the following XML AST.
\vspace{5mm}
\begin{verbatim}
<?xml version="1.0" encoding="UTF-8"?>
<SgSourceFile file="input.c" lang="2" fmt="2">
 <SgGlobal>
 ...omitted...
 </SgGlobal>
</SgSourceFile>
\end{verbatim}
\vspace{5mm}
ROSE uses C++ class objects, called \emph{Sage III class objects}, to
represent nodes of an AST.  That is, every node of a ROSE AST is an
object of a Sage III class. Each Sage III class is a subclass of the
\texttt{SgNode} class.  For example, the \texttt{SgGlobal} class is one
Sage III class, and its object represents the global scope of a code.
On the other hand, every node of an XML AST is an XML element whose name
is the same as the Sage III class name of its corresponding ROSE AST
node.  Hence, an \texttt{$<$SgGlobal$>$} element appears right after an
\texttt{$<$SgSourceFile$>$} element, which corresponds to an
\texttt{SgSourceFile} class object representing the source code.  See
the ROSE reference manual\cite{rose} for more details of Sage III
classes. Chapter~\ref{sec:attrib} also describes XML elements of Sage
III classes currently supported by XevXML.

XML attributes are used to keep the necessary information to rebuild
each ROSE AST. For example, \texttt{$<$SgSourceFile$>$} has a
\texttt{lang} attribute so that an XML AST can record the language of
the original source code.  The attributes of XML AST nodes are described
in Chapter~\ref{sec:attrib}.

To customize the format of an XML AST, XevXML offers two internal C++
classes, \texttt{XevSageVisitor} and \texttt{XevXmlVisitor}.  The former
class traverses an AST of Sage III classes used in ROSE, and translates
it to an XML AST.  The latter traverses an XML AST to rebuild ROSE's
AST.

The \texttt{XevConvertRoseToXml()} function internally uses the
\texttt{XevSageVisitor} class for converting a ROSE AST to an XML AST.
The \texttt{XevSageVisitor} class is a Vistor pattern class that visits
every node of a ROSE AST in a depth-first fashion. Whenever
\texttt{XevSageVisitor} visits an AST node, it writes an XML element
whose name is the same as the Sage III class name of an AST node, e.g.,
\texttt{SgGlobal}.  When writing XML attributes of each XML element,
\texttt{attribSg*()} method is invoked (\texttt{Sg*} is the name of a
Sage III class). Similarly, when writing the sub-nodes of each XML
element, \texttt{inodeSg*()} method is invoked.  Therefore, by
overloading those methods, a user can customize XML documents generated
by the \texttt{XevSageVisitor} class.

In the following code, a new attribute \texttt{"foo"} whose value is
\texttt{"bar"} is added to an \texttt{$<$SgGlobal$>$} element.
\begin{framed}
\begin{src}
/* custom.cpp */
#include <xevxml.hpp>

class MyVisitor: public XevXml::XevSageVisitor
{
public:
  MyVisitor():XevXml::XevSageVisitor() {}

  void attribSgGlobal(SgNode* node)
  {
    // print the default attributes
    XevXml::XevSageVisitor::attribSgGlobal(node);
    std::ostream& os = getOutputStream();

    os << " foo=\"str\"";
    os << " bar=\"1\"";
  }
};

int main(int argc,char** argv)
{
  SgProject* sageProject=frontend(argc,argv);
  MyVisitor visitor;
  XevXml::XevXmlOption opt;

  XevXml::XevInitialize();
  visitor.setXmlOption(&opt);
  visitor.write(std::cout,&sageProject);
  XevXml::XevFinalize();
  return 0;
}
\end{src}
\end{framed}

The \texttt{MyVisitor} class is a subclass of \texttt{XevSageVisitor}.
The \texttt{XevSageVisitor::setXmlOption()} method is used to pass an
\texttt{XevXmlOption} object to the \texttt{MyVisitor} class
object. After that, the \texttt{XevSageVisitor::write()} method is
invoked to write an XML AST to a given \texttt{std::ostream} object.

The \texttt{XevSageVisitor::attribSgGlobal()} method is overloaded by
\texttt{MyVisitor::attribSgGlobal()}.  In the new method, the reference
to an \texttt{std::ostream} object, which is given by the first argument
of the \texttt{XevSageVisitor::write()} method,
i.e.,~\texttt{std::cout}, is obtained by calling
\texttt{getOutputStream()}.  Then, two strings are passed to the object.
As a result, the strings appear as XML attributes of an
\texttt{$<$SgGlobal$>$} element.

\vspace{5mm}
\texttt{\% ./custom input.c}
\begin{verbatim}
<?xml version="1.0" encoding="UTF-8"?>
<SgSourceFile file="input.c" lang="2" fmt="2">
 <SgGlobal foo="str" bar="1">
 ...omitted...
 </SgGlobal>
</SgSourceFile>
\end{verbatim}
\vspace{5mm}

The \texttt{XevConvertXmlToRose()} function internally uses the
\texttt{XevXmlVisitor} class for converting an XML AST to a ROSE AST.
The \texttt{XevXmlVisitor} class is a Vistor pattern class that visits
every node of an XML AST in a depth-first fashion.  Whenever
\texttt{XevXmlVisitor} visits an XML element, \texttt{visigSg*()} method
is invoked. Therefore, by overloading such a method, a user can
customize the rebuilding process of an AST.

\begin{framed}
\begin{src}
/* custom2.cpp */
#include <xevxml.hpp>
#include <xmlutils.hpp>

class MyXmlVisitor: public XevXml::XevXmlVisitor
{
public:
  SgNode* visitSgGlobal(xercesc::DOMNode* node, SgNode* parent)
  {
    SgNode* n =
      XevXml::XevXmlVisitor::visitSgGlobal(node,parent);

    std::string sval;
    if(XevXml::XmlGetAttributeValue(node,"foo",&sval)==true){
      std::cerr << " attribute \"foo\" is found" << std::endl;
      std::cerr << " the value is " << sval << std::endl;
    }
    else {
      std::cerr << " attribute \"foo\" is not found" << std::endl;
    }

    int ival;
    if(XevXml::XmlGetAttributeValue(node,"bar",&ival)==true){
      std::cerr << " attribute \"bar\" is found" << std::endl;
      std::cerr << " the value is " << ival << std::endl;
    }
    else {
      std::cerr << " attribute \"bar\" is not found" << std::endl;
    }

    float fval;
    if(XevXml::XmlGetAttributeValue(node,"baz",&fval)==true){
      std::cerr << " attribute \"baz\" is found" << std::endl;
      std::cerr << " the value is " << fval << std::endl;
    }
    else {
      std::cerr << " attribute \"baz\" is not found" << std::endl;
    }

    return n;
  }
};

int main(int argc,char** argv)
{
  SgProject* sageProject=0;
  MyXmlVisitor visitor;
  XevXml::XmlInitialize();
  visitor.read(std::cin,&sageProject);
  XevXml::XmlFinalize();
  return 0;
}
\end{src}
\end{framed}

When an XML AST is traversed, every XML element is represented as an
\texttt{xercesc::DOMNode} class object. Because of the depth-first AST
traversal, its parent node is already converted to an \texttt{SgNode}
object in many cases.  The pointers to those two objects are given to
\texttt{XevXmlVisitor::visitSg*()} as function arguments.  If the
\texttt{SgNode} object of the parent node is not available yet, the
second argument is \texttt{NULL}.

\texttt{XevXmlVisitor::visitSg*()} has to return a pointer to an
\texttt{SgNode} object, which is actually an \texttt{Sg*} object.  For
example, the \texttt{XevXmlVisitor::visitSgGlobal()} method is called
when a \texttt{$<$SgGlobal$>$} element is visited.  This method and its
overloaded versions are expected to return a pointer to an
\texttt{SgGlobal} object.

In the above code, \texttt{XevXmlVisitor::visitSgGlobal()} is overloaded
by the \texttt{MyXmlVisitor} class so that a utility function,
\texttt{XevXml::XmlGetAttributeValue()}, is called for checking if each
attribute is given or not.  Here,
\texttt{XevXml::XmlGetAttributeValue()} is a C++ template function
defined in \texttt{xmlutils.hpp}.

\begin{verbatim}
% ./custom input.c |./custom2
 attribute "foo" is found
 the value is str
 attribute "bar" is found
 the value is 1
 attribute "baz" is not found
\end{verbatim}

\section{Summary}
XmlXML is an extensible code transformation framework. Its internal
structures and behaviors can be customized if necessary. The utility
functions and classes will be useful to develop tools for
transformation, visualization, and analysis of an XML AST. Use of XML
for representing an AST will by design be helpful to make those tools
interoperable.

%The interoperability is one important benefit of using XevXML.
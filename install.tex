
\chapter{Installation}\label{chap:app}


\section{Requirements}

\begin{itemize}
 \item ROSE compiler infrastructure -- \url{http://rosecompiler.org/}
 \item Apache Xerces C++ 3.1.1 -- \url{http://xerces.apache.org/}
 \item Apache Xalan C++ 1.0 -- \url{http://xml.apache.org/xalan-c/}
 \item PicoJSON -- \url{https://github.com/kazuho/picojson/}
\end{itemize}




\section{Installation guide}
\begin{enumerate}
 \item Fist of all, some environmental variables such as
       \texttt{LD\_LIBRARY\_PATH}, \texttt{JAVA\_HOME}, and \texttt{CXX}
       must be correctly set so as to use ROSE, Xerces, and Xalan. Use
       \texttt{ROSE\_HOME} and \texttt{PICOJSON\_HOME} for specifying
       the diretories of ROSE and PicoJSON, respectively.

       For example, if \texttt{picojson.h} is located in
       \texttt{/usr/local/include}, \texttt{PICOJSON\_HOME} needs be set
       to \texttt{/usr/local}.

 \item Create a new directory at the top diretocry for building the package.
       
       \texttt{\% mkdir mybuild}

       \texttt{\% cd mybuild}

 \item Run the \texttt{cmake} command to generate Makefile and copy necessary files.

       \texttt{\% cmake ../}

       The \texttt{cmake} command accepts various options. For example,
       the install diretory is changed by using the
       \texttt{-DCMAKE\_INSTALL\_PREFIX} option.

       \texttt{\% cmake -DCMAKE\_INSTALL\_PREFIX=/usr/local/xevxml ../}

       See the \texttt{cmake} manual for more details\cite{cmake}.
% \item Generate a configure script file using the autoconf command\\

%       \texttt{\% autoconf}

% \item Some environmental variables such as \texttt{LD\_LIBRARY\_PATH},
%       \texttt{JAVA\_HOME}, and \texttt{CXX} must be correctly set so as to
%       use ROSE, Xerces, and Xalan.

% \item Run the configure script\\
       
%       \texttt{\% ./configure --prefix=/path/to/somewhere $\backslash$}\\
%       \hspace{75pt}\texttt{--with-rosedir=/usr/local/    $\backslash$}\\
%       \hspace{75pt}\texttt{--with-xercescdir=/usr/local  $\backslash$}\\
%       \hspace{75pt}\texttt{--with-xalancdir=/usr/local}

 \item Run the GNU make command.

   To compile the package,\\
       \texttt{\% make}
       
   To install the package,\\
       \texttt{\% make install}

%   To test the package,\\
%       \texttt{\% make check}
\end{enumerate}

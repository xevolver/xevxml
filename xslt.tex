\chapter{AST Transformation Rules}\label{chap:xslt}

XevXML represents an AST in an XML format, and various XML tools are
hence available for AST transformation. In XevXML, the transformation
rules are (internally) represented as XSLT rules. If a special code
transformation is needed, the most flexible and expressive way is to
write an XSL rule for the transformation.

XevXML provides a lot of predefined rules that are also written in XSLT.
Users can customize those predefined rules for their own
purposes. XevXML also provides an XSLT rule template as a sample for
users to newly define their special transformation rules.

This chapter describes how to define and customize an XSLT rule for AST
transformation in XevXML.


\section{Predefined Rules}\label{sec:predef}
Remember that, in XevXML, every transformation rule is written in
XSLT. You can write such a rule from scratch if you want.  But, a code
transformation rule generally consists of many ``XSLT template rules,''
and some of them are reusable in other code transformation rules.
Therefore, XevXML offers some predefined XSLT template rules that can be
used as a part of a user-defined code transformation rule.

Let's get started with a simple XSLT rule before explaining the
predefined rules.
\begin{framed}
\begin{src}
<?xml version="1.0" encoding="UTF-8"?>
<xsl:stylesheet version="1.0"
   xmlns:xsl="http://www.w3.org/1999/XSL/Transform"
   xmlns:exslt="http://exslt.org/common">

  <xsl:template match="/">
    <xsl:apply-templates/>
  </xsl:template>

  <xsl:template match="*">
    <xsl:copy>
      <xsl:copy-of select="@*"/>
      <xsl:apply-templates/>
    </xsl:copy>
  </xsl:template>

  <xsl:template match="SgFortranDo">
    <xsl:if test="preceding-sibling::*[1]/SgPragma/@pragma='xev loop_tag'">
      !pragma is found
    </xsl:if>
    <xsl:copy>
      <xsl:copy-of select="@*"/>
      <xsl:apply-templates mode="loopbody"/>
    </xsl:copy>
  </xsl:template>

  <xsl:template match="*" mode="loopbody">
    ! in a loop body
    <xsl:copy>
      <xsl:copy-of select="@*"/>
      <xsl:apply-templates mode="loopbody"/>
    </xsl:copy>
  </xsl:template>
</xsl:stylesheet>
\end{src}
\end{framed}

As mentioned in Section~\ref{sec:xslt}, an XSLT rule usually consists of
several ``template'' rules expressed by \texttt{$<$xsl:template$>$}
elements. In the above example, three template rules are defined. Each
template rule has a \texttt{match} attribute whose value is an XPath
expression. When an XSLT rule is applied to an XML AST, every XML
element in the XML AST, i.e.~every AST node, is visited once in a
depth-first manner. Then, if an XML element matches the XPath expression
of the \texttt{match} attribute, the template rule is applied to the
matched XML element.

The first template rule matches the root node of an XML AST that is an
\texttt{$<$SgSourceFile$>$} element, and its rule is applied to the root
node.

The second template rule matches every node in the XML AST, and its rule
is applied to the node unless a more specific rule matches the node.

The third template rule matches only an \texttt{$<$SgFortranDo$>$}
element. This uses an \texttt{$<$xsl:if$>$} element to check if an XPath
expression is true. The expression is true only if
\begin{itemize}
 \item The preceding sibling element of the \texttt{$<$SgFortranDo$>$}
       element has an \texttt{$<$SgPragma$>$} element as a sub-node,
 \item The \texttt{$<$SgPragma$>$} element has a \texttt{pragma}
       attribute, and
 \item The attribute value is a string of "xev loop\_tag".
\end{itemize}
If all the conditions are met, the template rule inserts a comment,
``pragma is found,'' to the code before copying sub-nodes of the matched
\texttt{$<$SgFortranDo$>$} element.

In the third template rule, an \texttt{$<$xsl::apply-templates$>$}
element is used with a \texttt{mode} attribute. As a result, template
rules with the same mode are applied to the sub-nodes. In this example,
the fourth template rule is applied to every sub-node of a
\texttt{$<$SgFortranDo$>$} element, for which all the above conditions
are met. The fourth template rule instead of the second rule matches
every node if the rule is called with the \texttt{loopbody} mode. In
this way, different template rules can be applied only to a subset of
AST nodes.


If the third rule and/or the fourth rule is modified to transform a loop
structure and/or a loop body, the above XSLT rule can be used as a loop
transformation rule that is applied only to loops annotated with
\texttt{"xev loop\_tag"}. Although such a loop transformation rule could
be complex, it is reusable for many application codes.  Therefore,
XevXML provides a library of XSLT template rules, in which code
transformation rules required for basic loop optimization techniques are
predefined.

In the current version of XevXML, all predefined rules are written for
Fortran programs. It is ongoing to define such rules for C programs.
The predefined rules currently included in the library are as follows.

\begin{longtable}[l]{l|l|l}
 \hline
 Mode & Parameters & Description \\
 \hline\hline
 \endfirsthead
 \multicolumn{3}{l} {(continued)}\\
 \hline
 Mode & Parameters & Description \\
 \hline\hline
 \endhead
 \hline
 \multicolumn{3}{r} {(continue to next page)}\\
 \endfoot
 \hline
 \endlastfoot
 \multicolumn{3}{l}{Basic loop optimization rules}\\
 \hline

\texttt{xevLoopCollapse} & \texttt{firstLoop} & \multirow{3}{9cm}{Two
 loops are collapsed into a loop. Two parameters, \texttt{firstLoop} and
 \texttt{secondLoop}, specify the names of index variables of the loops
 to be collapsed. } \\
& \texttt{secondLoop} & \\ &&\\ \hline

\texttt{xevLoopFission} & \todo{none?}& \multirow{3}{9cm}{A loop is broken into
 multiple loops. Now each statement in the original loop is moved to a
 different loop. Thus, every new loop contains only one statement in its
 body.} \\
&&\\ &&\\ \hline

\texttt{xevLoopFusion} & \todo{none?}& \multirow{1}{9cm}{Two consecutive loops are
 fused into a single loop. Statements in the two loops are moved into
 the body of the new loop.} \\
 &&\\ &&\\ \hline

\texttt{xevLoopInterchange} & \todo{none?} & \multirow{3}{9cm}{\todo{Two
 consecutive loops are interchanged?}} \\
&&\\ &&\\ \hline

\texttt{xevLoopTile} & \texttt{loopName} & \multirow{4}{9cm}{\todo{A loop's
 iteration space is partitioned into blocks. \texttt{loopName} is
 the name of the index variable. \texttt{start} and \texttt{end} are the
 lower and upper bounds of the space. \texttt{size} is the block size.}} \\
& \texttt{start} & \\
& \texttt{end} &\\
& \texttt{size} &\\ \hline

\texttt{xevLoopUnroll} & \texttt{loopName} & \multirow{3}{9cm}{A loop is
 unrolled. \texttt{loopName} is the name of the index
 variable. \texttt{factor} is an unroll factor; every
 statement in the loop body is duplicated \texttt{factor} times.} \\ &
 \texttt{factor} & \\ &&\\ \hline

 \multicolumn{3}{l}{CHiLL-compatible versions of optimization rules}\\\hline

 \texttt{chill\_fuse} & \todo{none?} & \multirow{1}{9cm}{Two consecutive loops are
 fused into a single loop. Statements in the two loops are moved into
 the body of the new loop.} \\
 &&\\ &&\\ \hline

 \texttt{chill\_permute} & \texttt{firstLoop} & \multirow{1}{9cm}{The order of up to
 three loops are changed. \texttt{firstLoop}, \texttt{secondLoop}, and
 \texttt{thirdLoop} are the names of index variables used by the loops
 to be permuted.} \\
 &\texttt{secondLoop}&\\ &\texttt{thirdLoop}&\\ \hline

 \texttt{chill\_split} & \todo{none?} & \multirow{1}{9cm}{This is the
CHiLL-compatible version of \texttt{xevLoopFission}.} \\ \hline

\texttt{chill\_tile} & \texttt{loopName} & \multirow{4}{9cm}{\todo{A loop's
 iteration space is partitioned into blocks. \texttt{loopName} is
 the name of the index variable. \texttt{start} and \texttt{end} are the
 lower and upper bounds of the space. \texttt{size} is the block size.}} \\
& \texttt{start} & \\
& \texttt{end} &\\
& \texttt{size} &\\ \hline

\texttt{chill\_unroll} & \texttt{loopName} & \multirow{3}{9cm}{A loop in
 unrolled. \texttt{loopName} is the name of the index
 variable. \texttt{factor} is an unroll factor; every
 statement in the loop body is duplicated \texttt{factor} times.} \\ &
 \texttt{factor} & \\ &&\\ \hline

\texttt{chill\_unroll} & \texttt{loopName} & \multirow{4}{9cm}{The outer
 loop of a loop nest is unrolled. \texttt{loopName} is the name of the
 index variable. \texttt{factor} is an unroll factor; every statement in
 the loop body is duplicated \texttt{factor} times.} \\ &
 \texttt{factor} & \\ &&\\ &&\\  \hline
\end{longtable}

In the transformation rule library, a code transformation rule is
decomposed to three steps. One is initialization. Another is to move
onto an XML element that represents the root node of a subtree to be
transformed. The other is to transform the subtree, and write it in XML.
